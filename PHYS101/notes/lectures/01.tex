\lesson{Oct 4 2022}{Introduction}

\subsection{Physics as a tool}%
\label{sub:physics}
\info{This section is not needed, it is more of a conceptual overlook of physics as a science.}
\unsure{This is more of my personal rant than an educated opinion.}

It is frequently the case that we forget that science as a whole is no more than a mere model that tries to reflect what reality is through our own perspective. It does not represent the truth nor claim to. It tries its best to be a useful tool for the human kind to progress and improve their own lives.

Physics, therefore, is no exception. It is a collection of theories which failed to be proven wrong based on the empirical data we have. It is a tool that we use to understand the world around us and to predict the future; a \textit{tool}, not a \textit{truth}.

Thus no one should think too highly of physics as something holy and true in some absolute sense. Thus, we should think of our models as a distorted view of reality, and we should keep in mind that the hypothetical statements they make are not representative of the \textit{true} nature of \textit{existence} itself.

Debates on the true nature of existence is more of a philosophical thing. Physics, even if theoretical, is an experimental science, not humanities. It does not try to answer questions like dilemmas of god, free will, ideologies, etc. We only think of how we can improve our current theories to fit the reality we observe.

Furthermore, absurd theories (e.g string theory, quantum gravity, etc) are not profound physics theories. They are more of a set of arbitrary hypothesis bunched together to form a model that seem coherent yet does not new scarily conform to our reality. It is not to say that such theories are pseudoscience or they are not worth studying--even the theory of relativity was in such a state when it was first published. It is just that they are not \textit{real} within our studying of the physical nature. They are more like philosophical speculations of the world; they may be true, but, for the most part, they are unreliable perspectives.

\subsection{Formulation of a Mathematical Structure}
\label{sub_sec:sub_section_1}

Mathematical structures are made to help us, not to be enslaved to.
Their formulation is like a game; it consists of a hierarchy of:
\begin{enumerate}
    \item Goals
    \item Rules
    \item Definitions
\end{enumerate}

\begin{example}
    Cards Game
    \begin{itemize}
        \item Goals: winning (through gaining most cards).
        \item Rules:
            \begin{itemize}
                \item Joker takes all, number takes its combination, etc.
                    \item Poker
                    \item etc
            \end{itemize}
        \item Definitions: Ace of hearts, Knight of diamonds, etc.
    \end{itemize}
\end{example}

\begin{note}
    We can have the same game definitions with a different set of rules.
\end{note}

\begin{definition}[Logic]
    The process of using a set of definitions within a set of rules/constrains to reach a desired set of goals.
\end{definition}

\begin{note}
    Mathematical structures are not a truth within themselves; they are just tools to help us represent phenomenons in the real life efficiently.
\end{note}
\begin{example}
    Useful Mathematical Structures
    \begin{itemize}
        \item $\frac{1}{2}+\frac{1}{3}=\frac{5}{6}$ is a useful mathematical construct in some cases
            \begin{itemize}
                \item You worked half half an hour yesterday and worked third of an hour today, \textbf{then} you worked a total of 50 minutes out of 60 minutes.
                \item You won 1 out of 2 games yesterday and won 1 out of 3 games today, \textbf{then} you won a total of 2 out of 5 games
            \end{itemize}
        \item $\frac{1}{2}+\frac{1}{3}=\frac{2}{5}$ is a plausible construct in most cases, but--clearly--not in all cases.
    \end{itemize}
\end{example}

\begin{definition}[Truth]
    The combination of a set of \textit{definitions} and a set of \textit{rules}.
\end{definition}
\begin{definition}[Validity]
    The inescapable conclustions formed by a given set of truth.
\end{definition}

\begin{example}
    Truths
    \begin{itemize}
        \item $\vec{v}_{1}\times \vec{v}_{2}=n|\vec{v}_{1}||\vec{v}_{2}|\sin\theta$ (We define the cross product as the perpendicular vector of the area formed by a parallelogram whose sides are formed from the vectors since it is a useful physical quantity).
        \item $\vec{v}_{1}\times \vec{v}_{2}=-\vec{v}_{2}\times \vec{v}_{1}$ (We define the anti-commutative property of cross product since it represents rotational quantities).
    \end{itemize}
\end{example}
\begin{example}
    Validities
    \begin{itemize}
        \item $(\vec{v}_{1}\times \vec{v}_{2})\times \vec{v}_{3}\not=\vec{v}_{1}\times (\vec{v}_{2}\times \vec{v}_{3})$ (We can prove this using already defined proprieties and rules of the cross product).
    \end{itemize}
\end{example}
% subsection sub_section_1 (end)

\newpage
