%%%%%%%%%%%%%%%%%%%%%%%%%%%%% Define Article %%%%%%%%%%%%%%%%%%%%%%%%%%%%%%%%%%
\documentclass[stu]{apa7}
%%%%%%%%%%%%%%%%%%%%%%%%%%%%%%%%%%%%%%%%%%%%%%%%%%%%%%%%%%%%%%%%%%%%%%%%%%%%%%%

%%%%%%%%%%%%%%%%%%%%%%%%%%%%% Packages %%%%%%%%%%%%%%%%%%%%%%%%%%%%%%%%%%%%%%%%
\usepackage[style=apa,backend=biber]{biblatex}
\usepackage{multicol}
\usepackage{hyperref}
%%%%%%%%%%%%%%%%%%%%%%%%%%%%%%%%%%%%%%%%%%%%%%%%%%%%%%%%%%%%%%%%%%%%%%%%%%%%%%%

%%%%%%%%%%%%%%%%%%%%%%%%%%%%% Bibliography %%%%%%%%%%%%%%%%%%%%%%%%%%%%%%%%%%%%
\addbibresource{references.bib}
%%%%%%%%%%%%%%%%%%%%%%%%%%%%%%%%%%%%%%%%%%%%%%%%%%%%%%%%%%%%%%%%%%%%%%%%%%%%%%%

%%%%%%%%%%%%%%%%%%%%%%%%%%%%%%% Title & Author %%%%%%%%%%%%%%%%%%%%%%%%%%%%%%%%
\title{Case Study (1) - Theranos: Unveiling the Realities of a Failed
Healthcare Startup}
\author{SalahDin Ahmed Salh Rezk}
\affiliation{Zewail City of Science and Technology}
\course{SCH 264 --- Intro to Entrepreneurship and Small Management}
\professor{Dr. Ashraf Hafez Badawi}
\duedate{\today}

\abstract{
Theranos, founded by Elizabeth Holmes in 2003, aimed to revolutionize
healthcare through cheaper and faster blood tests using innovative technology.
However, investigations in 2015 revealed the technology's failure, leading to
fraud charges against Holmes and the company's downfall. This case study delves
into the ethical, legal, and regulatory issues stemming from Theranos' actions,
highlighting lessons learned and impacts on healthcare innovation and
entrepreneurship.
}

\shorttitle{THERANOS CASE STUDY}
\keywords{Theranos, Elizabeth Holmes, Healthcare, Innovation, Entrepreneurship}
%%%%%%%%%%%%%%%%%%%%%%%%%%%%%%%%%%%%%%%%%%%%%%%%%%%%%%%%%%%%%%%%%%%%%%%%%%%%%%%

\begin{document}
    \maketitle
    \tableofcontents

    \begin{multicols}{2}
    \section{Introduction}

    Theranos was a healthcare startup founded by Elizabeth Holmes in 2003. The
    company was based on the idea of providing a cheaper, faster, and more
    convenient alternative to traditional blood tests. The company claimed to
    have developed a revolutionary technology that could perform a wide range
    of tests using just a few drops of blood from a finger prick. However, in
    2015, a series of investigative reports by the Wall Street Journal revealed
    that the technology did not work as claimed, and that the company had been
    misleading investors, patients, and regulators. This led to the downfall of
    Theranos, and Elizabeth Holmes was charged with fraud and other crimes.

    In this case study, we will examine the rise and fall of Theranos, and
    explore the ethical, legal, and regulatory issues that arose from the
    company's actions. We will also discuss the lessons that can be learned
    from the Theranos scandal, and how it has impacted the healthcare industry
    as a whole. Additionally, we will consider the implications of the Theranos
    case for the future of healthcare innovation and entrepreneurship.

    \subsection{History}

    Elizabeth Holmes was born in 1984 in Washington, D.C. She was raised in
    Houston, Texas, and showed an early interest in science and technology. She
    was inspired by her uncle, who was an entrepreneur, and by her grandfather,
    who was a scientist. She enrolled at Stanford University in 2002 to study
    chemical engineering, and dropped out a year later to start Theranos. \textcite{inc}

    The company's technology was based on a device called the Edison, which was
    supposed to be able to perform a wide range of tests using just a few drops
    of blood from a finger prick. The company claimed that the tests were
    faster, cheaper, and more convenient than traditional blood tests, and that
    they could be done at home or in pharmacies. The company also claimed that
    the technology was accurate and reliable, and that it had been validated by
    independent experts. \textcite{Auletta_2014}

    Theranos quickly attracted a lot of attention and investment. The company
    raised over \$700 million from investors, and was valued at over \$9 billion
    at its peak. Elizabeth Holmes was hailed as a visionary and a genius, and was
    featured on the cover of magazines like Forbes and Fortune. She was also
    named one of the most influential people in the world by Time magazine. \textcite{forbes}

    However, in 2015, a series of investigative reports by the Wall Street
    Journal revealed that the technology did not work as claimed, and that the
    company had been misleading investors, patients, and regulators. The reports
    alleged that the tests were inaccurate and unreliable, and that the company
    had used traditional blood testing machines for most of its tests. The
    reports also alleged that the company had manipulated test results, and that
    it had failed to comply with regulatory requirements. \cite{wsj}

    The revelations led to a series of investigations by federal and state
    authorities, and to lawsuits by investors and patients. The company was
    forced to shut down its labs, and to void or correct thousands of test
    results. Elizabeth Holmes was charged with fraud and other crimes, and is
    currently awaiting trial. The company's valuation plummeted, and it was
    forced to lay off most of its employees and to close its offices. \cite{nyt}

    \subsection{Vision and Mission}

    Theranos' vision was to revolutionize healthcare by making blood testing
    faster, cheaper, and more convenient. The company's mission was to empower
    individuals to take control of their health by providing them with access to
    affordable and accurate tests. The company believed that its technology
    could help to detect diseases earlier, to monitor chronic conditions more
    effectively, and to personalize treatments based on individual needs. The
    company also believed that its technology could help to reduce healthcare
    costs, to improve patient outcomes, and to promote preventive care. \cite{theranos}

    The company's vision and mission were inspired by Elizabeth Holmes' personal
    experience with the healthcare system. She had a fear of needles, and had
    fainted during a blood test as a child. She believed that the experience was
    traumatic and unnecessary, and that it could have been avoided if the test
    had been done with just a few drops of blood. She also believed that the
    experience was emblematic of the problems with the healthcare system, and
    that it could be solved with the right technology. \cite{holmes}

    Employees and investors were attracted to Theranos' vision and mission
    because they believed that the company could make a real difference in
    people's lives. They were inspired by the idea of using technology to
    transform healthcare, and by the prospect of disrupting the industry. They
    were also motivated by the opportunity to work with a charismatic and
    visionary leader like Elizabeth Holmes, and by the promise of financial
    rewards. \cite{fortune}

    \section{Issues and Challenges}

    The Theranos scandal raised a number of ethical, legal, and regulatory
    issues, including honesty, integrity, transparency, accountability, fraud,
    conspiracy, negligence, and violation of laws and standards. The company's
    actions were unethical because they involved deception, fraud, and harm to
    stakeholders. The company's actions were illegal because they involved
    violation of laws and regulations. The company's actions were unethical and
    illegal because they violated the trust of stakeholders, and because they
    undermined the reputation of the healthcare industry. \cite{wsj, fortune, nyt}

    \subsection{Challenges}

    Theranos faced challenges in meeting their commitments due to inaccurate
    representations of their technology. The US FDA certified their herpes
    virus test as reliable in July, but it was later revealed that they used
    third-party equipment from Siemens without disclosure. This lack of
    transparency extended to patients, investors, and business partners
    regarding the actual capabilities of their testing. Following an FDA
    investigation in 2015, Theranos acknowledged test errors shortly after.
    Elizabeth Holmes, the founder, was reported to have fabricated trivial
    details like her whereabouts to employees, creating a culture of distrust.
    Many employees were let go or left due to questioning or criticism. Despite
    laboratory staff noting flaws in the "Edison" technology and submitting
    error reports, these concerns were disregarded by management. Instead of
    focusing on rectifying issues, only accurate data was highlighted, and
    incorrect conclusions were overlooked. \cite{wsj}

    \subsection{Ethical Issues}

    The Theranos scandal raised a number of ethical issues, including honesty,
    integrity, transparency, and accountability. The company's actions were
    unethical because they involved deception, fraud, and harm to patients,
    investors, and employees. The company misled investors by exaggerating the
    capabilities of its technology, and by concealing the fact that it was using
    traditional blood testing machines for most of its tests. The company also
    misled patients by providing inaccurate and unreliable test results, and by
    failing to comply with regulatory requirements. The company misled employees
    by creating a toxic work environment, and by pressuring them to engage in
    unethical and illegal activities. \cite{wsj}

    The company's actions were also unethical because they violated the trust of
    stakeholders, and because they undermined the reputation of the healthcare
    industry. The company's actions eroded the trust of investors, who lost
    billions of dollars as a result of the scandal. The company's actions eroded
    the trust of patients, who received inaccurate and unreliable test results
    that could have put their health at risk. The company's actions eroded the
    trust of employees, who were subjected to abuse and retaliation if they
    raised concerns about the technology. The company's actions eroded the trust
    of regulators, who were misled about the company's compliance with
    requirements and standards. \cite{fortune}

    \subsection{Legal Issues}
    
    The Theranos scandal also raised a number of legal issues, including fraud,
    conspiracy, and negligence. The company's actions were illegal because they
    involved deception, misrepresentation, and violation of laws and regulations.
    The company committed fraud by making false and misleading statements to
    investors, patients, and regulators. The company conspired to commit fraud by
    colluding with partners and suppliers to deceive stakeholders. The company
    was negligent by failing to exercise due care and diligence in the design,
    development, and testing of its technology. \cite{nyt}

    The company's actions violated a number of federal and state laws, including
    the Securities Act of 1933, the Securities Exchange Act of 1934, the Food,
    Drug, and Cosmetic Act, and the Health Insurance Portability and
    Accountability Act. The company's actions also violated a number of ethical
    and professional standards, including the American Medical Association Code
    of Medical Ethics, the American Society for Clinical Laboratory Science Code
    of Ethics, and the American Society for Clinical Pathology Code of Ethics.
    \cite{ama, clpmag, ascp}

    The company's actions resulted in a number of legal consequences, including
    criminal charges, civil lawsuits, and regulatory sanctions. Elizabeth Holmes
    was charged with wire fraud, conspiracy to commit wire fraud, and other crimes.
    The company was sued by investors for securities fraud, by patients for medical
    malpractice, and by employees for wrongful termination. The company was fined
    by regulators for violations of laboratory standards, and was banned from
    operating in the healthcare industry. \cite{wp}

    \section{Conclusions and Recommendations}

    The Theranos scandal has taught us a number of important lessons about
    entrepreneurship, innovation, and ethics. The scandal has shown us that
    success is not guaranteed, and that failure is always a possibility. It has
    shown us that innovation is not enough, and that execution is just as
    important. It has shown us that vision and mission are not enough, and that
    values and culture are just as important. It has shown us that leadership is
    not enough, and that integrity and accountability are just as important. It
    has shown us that trust is not enough, and that transparency and honesty are
    just as important.

    One may recommend that entrepreneurs and innovators should focus on solving
    real problems, and on creating real value. They should focus on building
    products and services that people need and want, and that they are willing
    to pay for. Not only should they focus on building products and services
    that work, but also on building products and services that are safe, reliable,
    and effective. 

    \end{multicols}

    \newpage
    \printbibliography[heading=bibintoc]

\end{document}
