\lesson{Oct 18 2022}{Cartesian coordinates}
\label{sub:cartesian}

\change{Add operations before this}

To use vectors in a more practical context, we put them onto the Cartesian coordinates where:
\begin{itemize}
    \item All vectors starts from the origin point of the coordinate system we chose.
    \item The tip of the vector lies inside the coordinate space we chose.
\end{itemize}

It is not physically helpful to use vectors starting from different origin points, and when this is needed, it helps to clarify the difference between various systems for calculations to be accurate.

In order to actually use the proprieties of the Cartesian coordinate, we have to associate the vector's tip with some point $(a,b)$. Sometimes vectors and points are used interchangeably. Whenever they are used in such manner, the physical context clarifies this confusion, but in some cases, they both are used within the same system where the distinction between them is needed.

\begin{note}
    Vectors are not points, neither are equivalent to them.
\end{note}

\subsubsection{Unit vectors}%
\label{ssub:unit-vectors}


To both avoid the confusion with points and make calculations more computable we use unit vectors. Unite vectors are like a reference point for a given system. Their physical value is completely arbitrary, although usually used with real life units (meter, km, mile, etc).

\begin{definition}[Unit vectors]
    Vectors of length 1, orthogonal to each other, and parallel to the axes of the coordinate system, such that:
    \begin{itemize}
        \item the tip of $e_1$ is on the point $(1,0,\ldots)$.
        \item the tip of $e_2$ is on the point $(0,1,\ldots)$.
            \\ \vdots
        \item the tip of $e_n$ is on the point $(0,0,\ldots,1)$.
    \end{itemize}
\end{definition}

We usually refer to the first unit vectors as:
\begin{itemize}
    \item $e_1 = \hat{i}$
    \item $e_2 = \hat{j}$
    \item $e_3 = \hat{k}$
\end{itemize}

$\hat{i}$ and $\hat{j}$ forms the relation $\vec{A}=a\hat{i}+b\hat{j}$.

\newpage

