\documentclass{zc-ust-hw}

\usepackage{lipsum}
\usepackage[]{caption} 
\usepackage[]{subcaption} 
\usepackage[]{pgfplots}

\newcommand*{\name}{SalahDin Ahmed (202201079)}
\newcommand*{\course}{Thermodynamics, Wave Motion and Optics (PHYS201)}
\newcommand*{\assignment}{Participation 5}

\begin{document}

\maketitle

Back in 1877, Thomas Edison opened the gate for manufacturing sound record by
inventing a phonogram that uses sound to etches grooves onto a wax cylinder.
The old phonogram spins at 33 1/3 rpm (rotations per min). Suppose, you are
asked to examine the diffraction grating effect by shining laser light onto the
record as shown the figure. If the laser light is of \(\lambda = 632.8\) nm, you put the
screen at a distance of 4 m away from the record and you observed an
interesting red dot on the screen separated by 21 mm each, as illustrated.

\begin{enumerate}
  \item Deduce the number of ridges per centimeter there are along the radius
    of the record.

      \begin{align}
        \lambda &= d\sin \theta  \\
        d &= \frac{\lambda}{\sin \theta } = \frac{\lambda}{\sin \left( \tan ^{-1} \left( \frac{x}{L} \right) \right)  } \\
        &= \frac{632.8\times 10^{-9}}{\sin \left( \tan ^{-1} \left(  \frac{21\times 10^{3} }{4.0} \right)  \right) }  \\
        &= 1.2\times 10^{-2}  \\
        \frac{1}{d} &= \frac{1}{1.2\times 10^{-2} } = 83 \text{ridges/cm}
      .\end{align}

  \item If you know that the record has only one song of 4.01 minutes and occupy
    16 mm on the record. In such case, deduce the number of ridges there are per
    cm.
    \begin{align}
      4.01 \times 33.3 &= 134 \\
      \frac{134}{1.6} &= 84 \text{ridges/cm}
    .\end{align}
\end{enumerate}

\end{document}
