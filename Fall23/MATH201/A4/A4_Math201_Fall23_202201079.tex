\documentclass{zc-ust-hw}

\usepackage{breqn}

\name{SalahDin Ahmed Salh Rezk}
\id{202201079}
\course{Linear Algebra (MATH 201)}
\assignment{Assignment 4}

\begin{document}

\maketitle

\begin{enumerate}
  \item Given that,
    \[
      A=\begin{pmatrix} 3&0&0\\0&4&1\\0&-1&2 \end{pmatrix} 
    \] 
    \begin{enumerate}
      \item Is 3 an eigenvalue of \( A \)? If so, find its corresponding eigenspace.  
        \begin{sol}
          \begin{align}
            \det(A - \lambda I) &= 0 \\
            \det\begin{pmatrix} 3-\lambda & 0 & 0 \\ 0 & 4-\lambda & 1 \\ 0 & -1 & 2-\lambda \end{pmatrix} &= 0 \\
            (3-\lambda)((4-\lambda)(2-\lambda) - (-1)(-1)) &= 0 \\
            (3-\lambda)(\lambda^2 - 6\lambda + 9) &= 0 \\
            (3-\lambda)(\lambda - 3)^2 &= 0 \\
            \lambda &= 3
          .\end{align}
          \end{sol}
          \begin{sol}
          \begin{align}
            A - 3I &= \begin{pmatrix} 0 & 0 & 0 \\ 0 & 1 & 1 \\ 0 & -1 & -1 \end{pmatrix} \\
            R_2 \leftarrow R_2 + R_3
            \quad
            &\begin{pmatrix} 0 & 0 & 0 \\ 0 & 0 & 0 \\ 0 & -1 & -1 \end{pmatrix} \\
            R_3 \leftarrow -R_3
            \quad
            &\begin{pmatrix} 0 & 0 & 0 \\ 0 & 0 & 0 \\ 0 & 1 & 1 \end{pmatrix} \\
            R_1 \leftrightarrow R_3
            \quad
            &\begin{pmatrix} 0 & 1 & 1 \\ 0 & 0 & 0 \\ 0 & 0 & 0 \end{pmatrix}
          .\end{align}
          \begin{equation}
            \therefore \text{Eigenspace}= \text{Span}\left\{ \begin{pmatrix} 1\\0\\0 \end{pmatrix}, \begin{pmatrix} 0\\-1\\1 \end{pmatrix}  \right\}
          .\end{equation}
        \end{sol}
      \item Express, if possible, \( e^{A}  \) as a linear combination of \(
        I_{3} \), \( A \), \( A^2 \). 
        \begin{sol}
          \begin{align}
            e^{A} &= \alpha_{0} I_{3} + \alpha_{1} A + \alpha_{2} A^2 \\ 
            e^{\lambda} &= \alpha_{0} + \alpha_{1} \lambda + \alpha_{2} \lambda^2 \\ 
            \frac{d}{d\lambda} e^{\lambda} &= \alpha_{1} + 2\alpha_{2} \lambda \\
            \frac{d^2}{d\lambda^2} e^{\lambda} &= 2\alpha_{2}
          .\end{align}
          \begin{align}
            e^3 &= 2\alpha_{2} \implies \alpha_{2} = \frac{e^3}{2} \\ 
            e^{3} &= \alpha_{1} + 3e^3 \implies \alpha_{1} = -2e^3 \\
            e^{3} &= \alpha_{0} -6e^3 + \frac{9e^3}{2} \implies \alpha_{0} = \frac{5e^3}{2}
          .\end{align}
          \begin{equation}
            \therefore e^{A} = \frac{5e^3}{2} I_{3} - 2e^3 A + \frac{e^3}{2} A^2
          .\end{equation}
        \end{sol}
      \item Find, if possible, an orthogonal matrix \( P \), such that \( A=PDP^{T}  \). 
        \begin{sol}
          \begin{align}
            &\because \text{dim}(\text{Eigenspace}(A)) < \text{dim}(A) \implies
            A \text{ is not diagonalizable} \\
            &\therefore \nexists P \text{ such that } A = PDP^T
          .\end{align}
        \end{sol}
    \end{enumerate}

    \newpage

  \item Use Cayley-Hamilton theorem to find the exponential matrix \( e^{At}
    \) such that: 
    \[
      A=\begin{pmatrix} -6&-11&16\\2&5&-4\\-4&-5&10 \end{pmatrix} 
    .\] 

    \begin{sol}
      \begin{align}
        |A - \lambda I| &= 0 \\
        \left|\begin{array}{ccc} -6-\lambda & -11 & 16 \\ 2 & 5-\lambda & -4 \\ -4 & -5 & 10-\lambda \end{array}\right| &= 0
      .\end{align}
      \begin{align*}
        \lambda^3 - \text{Tr}(A)\lambda^2 + \text{Tr}(\text{Adj}(A))\lambda - |A| &= 0 \\
        \lambda^3 - 9\lambda^2 + 26\lambda - 24 &= 0
      .\end{align*}
      \begin{equation}
        \lambda = \{2, 3, 4\}
      .\end{equation}
      \begin{align}
        e^{\lambda t} = \alpha_{0} + \alpha_{1} \lambda + \alpha_{2} \lambda ^2 
        & \implies
        \begin{cases}
          e^{2t} &= \alpha_{0} + 2\alpha_{1} + 4\alpha_{2} \\
          e^{3t} &= \alpha_{0} + 3\alpha_{1} + 9\alpha_{2} \\
          e^{4t} &= \alpha_{0} + 4\alpha_{1} + 16\alpha_{2}
        \end{cases} \\
        &\implies
        \left[
          \begin{array}{ccc|c}
            1 & 2 & 4 & e^{2t} \\
            1 & 3 & 9 & e^{3t} \\
            1 & 4 & 16 & e^{4t}
          \end{array} 
        \right] \\
        &\implies 
        \begin{cases}
          \alpha_{2} &= \frac{1}{2}(e^{4t}-2e^{3t}+e^{2t}) \\
          \alpha_{1} &= -\frac{5}{2}e^{4t} + 6e^{3t} - \frac{7}{2}e^{2t} \\
          \alpha_{0} &= 3e^{4t}-8e^{3t}+6e^{2t}   
        \end{cases}
      .\end{align}
      \begin{align}
        e^{At} &= \alpha_{0} I_{3} + \alpha_{1} A + \alpha_{2} A^2 \\
        e^{At} &= (3e^{4t}-8e^{3t}+6e^{2t}) I_{3} + (-\frac{5}{2}e^{4t} + 6e^{3t} - \frac{7}{2}e^{2t}) A + (\frac{1}{2}(e^{4t}-2e^{3t}+e^{2t})) A^2
      .\end{align}
      \begin{equation}
        A^2 = \begin{bmatrix} 
          -50&-69&108\\
          14&23&-28\\
          -26&-31&56
        \end{bmatrix} 
      \end{equation}
      \begin{equation}
        \begin{split}
          e^{At} &= (3e^{4t}-8e^{3t}+6e^{2t}) \begin{bmatrix}
            1&0&0\\0&1&0\\0&0&1 \end{bmatrix} + (-\frac{5}{2}e^{4t} + 6e^{3t} -
            \frac{7}{2}e^{2t}) \begin{bmatrix} -6&-11&16\\2&5&-4\\-4&-5&10
              \end{bmatrix} \\ &+ (\frac{1}{2}(e^{4t}-2e^{3t}+e^{2t}))
              \begin{bmatrix} -50&-69&108\\ 14&23&-28\\ -26&-31&56
              \end{bmatrix}
        \end{split}
      \end{equation}
      \begin{equation}
        = \begin{bmatrix} 
          -7e^{4t}+6e^{3t}+e^{2t}&-7e^{4t}+3e^{3t}+4e^{2t}&14e^{4t}-12e^{3t}-2e^{2t}\\
          2e^{4t}-2e^{3t}&2e^{4t}-e^{3t}&-4e^{4t}+4e^{3t}\\
          -3e^{4t}+2e^{3t}+e^{2t}&-3e^{4t}+e^{3t}+2e^{2t}&6e^{4t}-4e^{3t}-e^{2t}
        \end{bmatrix} 
      .\end{equation}
    \end{sol}

    \newpage

  \item Given that:
    \[
      A=\begin{bmatrix} 2&-2&-4\\-1&3&4\\1&-2&-3 \end{bmatrix} 
    .\] 

    \begin{enumerate}
      \item Show that \( A \) is an idempotent matrix (i.e. \( AA=A \)).
        \begin{sol}
          \begin{equation}
            A A = A \iff \lambda = 0 \lor \lambda = 1
          .\end{equation}
          \begin{align}
            |A - \lambda I| &= 0 \\
            \left|\begin{array}{ccc} 2-\lambda & -2 & -4 \\ -1 & 3-\lambda & 4 \\ 1 & -2 & -3-\lambda \end{array}\right| &= 0 \\
            \lambda^3 - \text{Tr}(A)\lambda^2 + \text{Tr}(\text{Adj}(A))\lambda - |A| &= 0 \\
            \lambda^3 - 2\lambda^2 +\lambda &= 0 \\
            \lambda &= \{0, 1, 1\}
          .\end{align}
          \begin{equation}
            \therefore A A = A
          \end{equation}
        \end{sol}
      \item Use part (a) to solve the initial value problem.
        \begin{align*}
          x'&=2x-2y-4z \\
          y'&=-x+3y+4z \\
          z'&=x-2y-3z
        \end{align*}
        Such that, \( x(0)=y(0)=z(0)=1 \)
        \begin{sol}
          \begin{align}
            e^{At} &= \alpha_{0} I_{3} + \alpha_{1} A + \alpha_{2} A^2 \\
            \intertext{Since \(A\) is idempotent i.e. \(A^2=A\),}
                   &= \alpha_{0} I_{3} + \alpha_{1} A \\
           \intertext{Using Cayley-Hamilton:}
            e^{\lambda t} &= \alpha_{0} + \alpha_{1} \lambda \\
            e^{0} &= \alpha_{0} + \alpha_{1} \times 0 \implies \alpha_{0} = 1 \\ 
            e^{t} &= 1 + \alpha_{1} \times 1 \implies \alpha_{1} = e^{t} - 1 \\
          .\end{align}
          \begin{align}
            e^{At} &= I_{3} + (e^{t}-1)A \\
            &= \begin{bmatrix} 1&0&0\\0&1&0\\0&0&1 \end{bmatrix} + (e^{t}-1) \begin{bmatrix} 2&-2&-4\\-1&3&4\\1&-2&-3 \end{bmatrix} \\
            &= \begin{bmatrix} 1+2(e^{t}-1)&-2(e^{t}-1)&-4(e^{t}-1)\\-1(e^{t}-1)&1+3(e^{t}-1)&4(e^{t}-1)\\1(e^{t}-1)&-2(e^{t}-1)&1+-3(e^{t}-1) \end{bmatrix} \\
            &= \begin{bmatrix} 2e^{t}-1&-2e^{t}+2&-4e^{t}+4\\-e^{t}+1&3e^{t}-2&4e^{t}-4\\e^{t}-1&-2e^{t}+2&-3e^{t}+4 \end{bmatrix}
          .\end{align}
          \begin{align}
            r &= e^{At}r_{0} \\
            &= \begin{bmatrix} 2e^{t}-1&-2e^{t}+2&-4e^{t}+4\\-e^{t}+1&3e^{t}-2&4e^{t}-4\\e^{t}-1&-2e^{t}+2&-3e^{t}+3 \end{bmatrix} \begin{bmatrix} 1\\1\\1 \end{bmatrix} \\
            &= \begin{bmatrix} -4e^{t}+5\\6e^{t}-5\\-4e^{t}+5 \end{bmatrix}
          .\end{align}
          \begin{align}
            x(t) &= -4e^{t}+5 \\
            y(t) &= 6e^{t}-5 \\
            z(t) &= -4e^{t}+5
          .\end{align}
        \end{sol}
    \end{enumerate}

  \item \,
    \begin{enumerate}
      \item Let \(A\) be a diagonalizable \(n\times n\)  matrix show that if the
        multiplicity of an eigenvalue \(\lambda\) is \(n\), then \(A=\lambda I\). 
        \begin{sol}
          \begin{align}
            A&= PD_{\lambda }P^{-1} \\
             &= P(\lambda I)P^{-1} \\
             &= \lambda PIP^{-1} \\
             &= \lambda I
          .\end{align}
        \end{sol}
      \item Use part (a) to show that the matrix \( \begin{bmatrix} 3&1\\0&3 \end{bmatrix}  \) is not diagonalizable. 
        \begin{sol}
          \begin{align}
            |A-\lambda I| &= 0 \\
            (3-\lambda)^2&= 0 \\
                         &\implies \lambda_{1}=\lambda_{2}=3
          .\end{align}
          Using (a), \(3I\) is the only matrix of multiplicity 2 that is
          diagonalizable
          \begin{gather}
            \because \begin{bmatrix} 3&1\\0&3 \end{bmatrix} \neq \begin{bmatrix} 3&0\\0&3 \end{bmatrix} \\
            \therefore \begin{bmatrix} 3&1\\0&3 \end{bmatrix} \text{ is not diagonalizable}
          .\end{gather}
        \end{sol}
    \end{enumerate}

    \newpage

  \item \, 
    \begin{enumerate}
      \item The eigenvalues (\(\lambda\))  of a diagonalizable square matrix \(A\) together
        with their multiplicities are given in the following table. 
        \begin{table}[H]
        \begin{center}
          \begin{tabular}[c]{|c|c|}
            \hline
            \(\lambda\) & \textbf{Multiplicity} \\
            \hline
            1 & 1 \\
            2 & 3 \\
            3 & 2 \\
            \hline
          \end{tabular}
        \end{center}
       \end{table}

       \begin{enumerate}
         \item Find the dimension of \( A \) 
           \begin{sol}
            \begin{align}
              \text{dim}(A) &= \sum_{k=1}^{n} \text{Mult}(\lambda_{k}) \\
              &= 1+3+2 \\
              &= 6
            .\end{align}
           \end{sol}
         \item Find the Trace and determinant of \( A \). 
           \begin{sol}
            \begin{align}
              \text{Tr}(A) &= \sum_{k=1}^{n} \lambda_{k} \\
              &= 1+3\times 3 + 3 \times 2 \\
              &= 13
            .\end{align}
            \begin{align}
              |A| &= \prod_{k=1}^{n} \lambda_{k} \\
              &= 1 \times  2^3 \times  3^2 \\
              &= 72
            .\end{align}
           \end{sol}
         \item Find the characteristic polynomial of \( A \). 
           \begin{sol}
            \begin{align}
              (\lambda-1)(\lambda-2)^3(\lambda-3)^2=0
            .\end{align}
           \end{sol}
         \item Find \(|e^{At}|\). 
           \begin{sol}
            \begin{align}
              |e^{At}| &= |PD_{e^{\lambda t}}P^{-1}| \\
              &= |P| |D_{e^{\lambda t}}| |P^{-1}|  \\
              &= |D_{e^{\lambda t}}| \\
              &= e^{\sum_{k=1}^{n} \lambda_{k}t} \\
              &= e^{\text{Tr}(A)t} \\
              &= e^{13t} 
            .\end{align}
           \end{sol}
       \end{enumerate}

     \item Solve the differential equations
       \[
         \frac{dy}{dt} = 2x-3y; \frac{dx}{dt} = 3x-4y
       .\]
       for \( x(0)=1;y(0)=0 \)
       \begin{sol}
         \begin{align}
           |A-\lambda I| &= 0 \\
           \left|\begin{array}{ccc}
             2-\lambda & -3 \\ -3 & 4-\lambda \\
           \end{array}\right| &= 0 \\
           -(3-\lambda)(3+\lambda)+8 &= 0 \\
           \lambda &= \pm 1
         .\end{align}
         \begin{align}
          \text{At \(\lambda = 1\)}
          &\quad
          \left[\begin{array}{cc|c}
              2&-4&0\\
              2&-4&0
          \end{array}\right]
          \sim
          \left[\begin{array}{cc|c}
              1&-2&0\\
              0&0&0
          \end{array}\right]
          \implies
          v_{1} = \begin{bmatrix} 2\\1 \end{bmatrix} \\
          \text{At \(\lambda = -1\)}
          &\quad
          \left[\begin{array}{cc|c}
              4&-4&0\\
              2&-2&0
          \end{array}\right]
          \sim
          \left[\begin{array}{cc|c}
              1&-1&0\\
              0&0&0
          \end{array}\right]
          \implies
          v_{1} = \begin{bmatrix} 1\\1 \end{bmatrix}
        .\end{align}
        \begin{gather}
          e^{At}r_{0} \\ 
          e^{At} \begin{bmatrix} 1\\0 \end{bmatrix} \\
          P D_{e^{\lambda t}} P^{-1} \begin{bmatrix} 1\\0 \end{bmatrix} \\
          \begin{bmatrix} 2&1\\1&1 \end{bmatrix} \begin{bmatrix} e^{t}&0\\0&e^{-t} \end{bmatrix} \begin{bmatrix} 1&-1\\-1&2 \end{bmatrix} \begin{bmatrix} 1\\0 \end{bmatrix} \\
          \begin{bmatrix} 2e^{t}-e^{-t} & -2e^{t}+2e^{-t}\\ e^{t}-e^{-t}&-e^{t}+2e^{-t} \end{bmatrix} \begin{bmatrix} 1\\0 \end{bmatrix} \\
          \begin{bmatrix} 2e^{t}-e^{-t}\\ e^{t}-e^{-t} \end{bmatrix} 
        .\end{gather}
        \begin{align}
          x(t) &= 2e^{t}-e^{-t} \\
          y(t) &= e^{t}-e^{-t}
        \end{align}
      \end{sol}
  \end{enumerate}

\end{enumerate}

\end{document}
