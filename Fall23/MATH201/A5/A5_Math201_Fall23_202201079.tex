\documentclass{zc-ust-hw}

\usepackage{breqn}

\name{SalahDin Ahmed Salh Rezk}
\id{202201079}
\course{Linear Algebra (MATH 201)}
\assignment{Assignment 5}

\begin{document}

\maketitle

\begin{enumerate}
  \item Let \( P_{2} \) denotes the vector space of all polynomials of degree at most 2.
    Consider the following inner product on \( P_{2} \), 
    \[
      \langle p,q \rangle = \int_{-1}^{1} p(x)q(x)dx.
    .\]
    \begin{enumerate}
      \item Find an orthogonal basis for \(P_{2}\). 
        \begin{sol}
          Consider the basis \(\{1, x, x^2\}\):
          \begin{equation}
            u_{1} = 1 \quad u_{2} = x \quad u_{3} = x^2
          .\end{equation}
          \begin{align}
            v_{1} &= u_{1} = 1 \\
            v_{2} &= u_{2} - \frac{\langle u_{2}, v_{1} \rangle}{\langle v_{1}, v_{1} \rangle}v_{1} = x - \frac{\int_{-1}^{1} xdx}{\int_{-1}^{1} 1dx} = x-0 = x \\
            v_{3} &= u_{3} - \frac{\langle u_{3}, v_{1} \rangle}{\langle v_{1}, v_{1} \rangle}v_{1} - \frac{\langle u_{3}, v_{2} \rangle}{\langle v_{2}, v_{2} \rangle}v_{2} \\
                  &= x^2 - \frac{\int_{-1}^{1} x^2dx}{\int_{-1}^{1} 1dx} - \frac{\int_{-1}^{1} x^3dx}{\int_{-1}^{1} x^2dx}x = x^2 - \frac{1}{3} - 0 = x^2 - \frac{1}{3}
          .\end{align}
          \begin{equation}
            \{v_{1}, v_{2}, v_{3}\} = \{1, x, x^2 - \frac{1}{3}\}
          \end{equation}
        \end{sol}
      \item Find the polynomial \( \hat{p}(t) \in P_{2} \) which best approximates the function
        \( f(t)=t^3 \) on $[ -1, 1 ]$. 
        \begin{sol}
          \begin{align}
          .\end{align}
          \begin{equation}
            \hat{p}(t) = \frac{3}{5}t
          \end{equation}
        \end{sol}
    \end{enumerate}

    \newpage
    
  \item The total revenue (in millions of dollars) of a certain company from
    2015 to 2018 are shown below, 
    \begin{table}[H]
      \begin{center}
        \begin{tabular}{|c|c|c|c|c|}
          \hline
          $x$ (Year 20-) & 15 & 16 & 17 & 18 \\
          \hline
          $y$ (Revenue) & 74 & 78 & 87 & 94 \\
          \hline
        \end{tabular}
      \end{center}
    \end{table}
    \begin{enumerate}
      \item Find the least squares regression line that best fits the data
        (Hint: use 2 digits only for the year variable \(x\), i.e. 15, 16, \dots). 
        \begin{sol}
          \begin{equation}
            y = \begin{bmatrix} 74\\78\\87\\94 \end{bmatrix} 
            \quad
            X = \begin{bmatrix} 1 & 15 \\ 1 & 16 \\ 1 & 17 \\ 1 & 18 \end{bmatrix}
            \quad
            \beta = \begin{bmatrix} b \\ m \end{bmatrix}
          .\end{equation}
          \begin{align}
            y &= X\beta \\
            \beta &= X^+y \\
            &= (X^TX)^{-1}X^Ty
          .\end{align}
          \begin{align}
            X^TX &= \begin{bmatrix} 4 & 66 \\ 66 & 1094 \end{bmatrix} \\
            (X^TX)^{-1} &= \begin{bmatrix} 54.7 & -3.3 \\ -3.3 & 0.2 \end{bmatrix} \\
            (X^TX)^{-1}X^T &= \begin{bmatrix} 5.2 & 1.9 & -1.4 & -4.7 \\ -0.3 & -0.1 & 0.1 & 0.3 \end{bmatrix} \\
            (X^TX)^{-1}X^Ty &= \begin{bmatrix}-30.6\\6.9\end{bmatrix}
          .\end{align}
          \begin{equation}
            b = -30.6 \quad m = 6.9 \implies y = 6.9x - 30.6
          .\end{equation}
        \end{sol}
      \item Use the above model to predict the total revenue in 2019.
        \begin{sol}
          \begin{equation}
            y = 6.9(19) - 30.6 = 100.5
          .\end{equation}
        \end{sol}
    \end{enumerate}

    \newpage

  \item \[
      A= \begin{pmatrix} 1&1\\0&1\\1&1 \end{pmatrix} \quad b= \begin{pmatrix} -1\\1\\1 \end{pmatrix}
  .\] 
    \begin{enumerate}
      \item Prove that the least squares solution for \(Ax=b\) is given by the
        normal equations: \(A^TAx=A^Tb\).
        \begin{sol}
          \begin{align}
            A^TAx &= A^Tb \\
            \begin{bmatrix} 1 & 0 & 1 \\ 1 & 1 & 1 \end{bmatrix} \begin{bmatrix} 1 & 1 \\ 0 & 1 \\ 1 & 1 \end{bmatrix} x &= \begin{bmatrix} 1 & 0 & 1 \\ 1 & 1 & 1 \end{bmatrix} \begin{bmatrix} -1 \\ 1 \\ 1 \end{bmatrix} \\
            \begin{bmatrix}2&2\\2&3\end{bmatrix} x &= \begin{bmatrix} 0 \\ 2 \end{bmatrix} \\
            \begin{bmatrix} 2 & 1 \\ 1 & 2 \end{bmatrix}^{-1} \begin{bmatrix} 2 & 1 \\ 1 & 2 \end{bmatrix} x &= \begin{bmatrix} 2 & 1 \\ 1 & 2 \end{bmatrix}^{-1} \begin{bmatrix} 0 \\ 2 \end{bmatrix} \\
            x &= \begin{bmatrix} -\frac{2}{3}\\ \frac{4}{3} \end{bmatrix} \label{eq:1}
          .\end{align}
          \begin{align}
            Ax &= b \\
            \tilde{x} &= A^+b \\
            &= \begin{bmatrix} 1 & 1 \\ 0 & 1 \\ 1 & 1 \end{bmatrix}^+ \begin{bmatrix} -1 \\ 1 \\ 1 \end{bmatrix} \\
            &= \begin{bmatrix}0.5&-1&0.5\\0&1&0\end{bmatrix} \begin{bmatrix} -1 \\ 1 \\ 1 \end{bmatrix} \\
            &= \begin{bmatrix} -\frac{2}{3}\\ \frac{4}{3} \end{bmatrix} \label{eq:2}
          \end{align}
          From \eqref{eq:1} and \eqref{eq:2} we can see that the least squares solution for \(Ax=b\) is given by the normal equations: \(A^TAx=A^Tb\) for the given cases of $A$ and $b$.
        \end{sol}
      \item Find the least squares solution for \(Ax=b\).
        \begin{sol}
          From the previous part:
          \begin{align}
            x &= \begin{bmatrix} -\frac{2}{3}\\ \frac{4}{3} \end{bmatrix}
          .\end{align}
        \end{sol}
      \item Find the closest vector to \(b\) in Col\(A\).
        \begin{sol}
          By definition, the closest vector to \(b\) in Col\(A\) is the least squares solution for \(Ax=b\). From the previous part:
          \begin{align}
            x &= \begin{bmatrix} -\frac{2}{3}\\ \frac{4}{3} \end{bmatrix}
          .\end{align}
        \end{sol}
    \end{enumerate}

    \newpage

  \item \begin{enumerate}
      
      \item Consider the inner product, \[
        \langle p,q \rangle = p(t_{0})q(t_{0})+ p(t_{1})q(t_{1}) + \ldots + p(t_{n})q(t_{n})
      .\] defined over \(P_{n}\) (the vector space of all polynomials of degree
      at most \(n\)). Let  \(\{1,t,t^2\}\)  be the standard basis of \(P_{2}\)
      and let \( t_{0}=-1,t_{1}=0,t_{2}=1 \), find an orthonormal basis for \(P_{2}\).
      \begin{sol}
        \begin{equation}
          u_{1} = 1 \quad u_{2} = t \quad u_{3} = t^2
        .\end{equation}
        \begin{align}
          v_{1} &= u_{1} = 1 \\
          v_{2} &= u_{2} - \frac{\langle u_{2}, v_{1} \rangle}{\langle v_{1}, v_{1} \rangle}v_{1} = t - \frac{t_{0}^2 + t_{1}^2 + t_{2}^2}{t_{0}^2 + t_{1}^2 + t_{2}^2} = t - 0 = t \\
          v_{3} &= u_{3} - \frac{\langle u_{3}, v_{1} \rangle}{\langle v_{1}, v_{1} \rangle}v_{1} - \frac{\langle u_{3}, v_{2} \rangle}{\langle v_{2}, v_{2} \rangle}v_{2} \\
                &= t^2 - \frac{t_{0}^3 + t_{1}^3 + t_{2}^3}{t_{0}^2 + t_{1}^2 + t_{2}^2} - \frac{t_{0}^4 + t_{1}^4 + t_{2}^4}{t_{0}^4 + t_{1}^4 + t_{2}^4}t = t^2 - 0 - 0 = t^2
        .\end{align}
        \begin{equation}
          \{v_{1}, v_{2}, v_{3}\} = \{1, t, t^2\}
        \end{equation}
      \end{sol}

    \item 
    \end{enumerate}
  
\end{enumerate}

\end{document}
