\documentclass{zc-ust-hw}

\usepackage{lipsum}
\usepackage[]{enumitem}

\newcommand*{\name}{SalahDin Rezk}
\newcommand*{\id}{202201079}
\newcommand*{\course}{MATH 201\\Linear Algebra and Vector Geometry}
\newcommand*{\assignment}{Assignment 2}

\begin{document}

\maketitle


%%%%%%%%%%%%%%%%%%%%
\section*{Question (1)}
%%%%%%%%%%%%%%%%%%%%

\begin{enumerate}[label=1.\arabic*]
  \item Find the determinant by row reduction to echelon form.
    \begin{enumerate}[label=\roman*.]
      \item $\begin{vmatrix}
          1 & 5 & -6 \\
          1 & 6 & 5 \\
          -2 & -8 & 7
        \end{vmatrix}$
        \begin{align}
          \begin{vmatrix}
            1 & 5 & -6 \\
            1 & 6 & 5 \\
            -2 & -8 & 7
          \end{vmatrix}
          &=
          \begin{vmatrix}
            1 & 5 & -6 \\
            0 & 1 & 11 \\
            0 & 2 & -5
          \end{vmatrix}
          \\
          &=
          \begin{vmatrix}
            1 & 5 & -6 \\
            0 & 1 & 11 \\
            0 & 0 & -27
          \end{vmatrix}
          \\
          &= 1 \times 1 \times (-27) = -27
        .\end{align}
      \item $\begin{vmatrix}
          1  & -2 & -1 & 0  & -2 \\
          0  & 3  & 8  & 5  & -7 \\
          -2 & 4  & 2  & 5  & -1 \\
          1  & -5 & 5  & -9 & -7 \\
          0  & 3  & 8  & 10 & 4
        \end{vmatrix}$
        \begin{align}
          \begin{vmatrix}
            1  & -2 & -1 & 0  & -2 \\
            0  & 3  & 8  & 5  & -7 \\
            -2 & 4  & 2  & 5  & -1 \\
            1  & -5 & 5  & -9 & -7 \\
            0  & 3  & 8  & 10 & 4
          \end{vmatrix}
        .\end{align}
        \begin{align}
          R_3 + 2R_1 \rightarrow R_3
          &\quad
          \begin{vmatrix}
            1  & -2 & -1 & 0  & -2 \\
            0  & 3  & 8  & 5  & -7 \\
            0  & 0  & 0  & 5  & -5 \\
            1  & -5 & 5  & -9 & -7 \\
            0  & 3  & 8  & 10 & 4
          \end{vmatrix}
          \\
          R_4 - R_1 + R_2 \rightarrow R_4
          &\quad
          \begin{vmatrix}
            1 & -2 & -1 & 0  & -2  \\
            0 & 3  & 8  & 5  & -7  \\
            0 & 0  & 0  & 5  & -5  \\
            0 & 0  & 14 & -4 & -12 \\
            0 & 3  & 8  & 10 & 4
          \end{vmatrix}
        \end{align}
        \begin{align}
          R_5 - R_2 \rightarrow R_5
          &\quad
          \begin{vmatrix}
            1 & -2 & -1 & 0  & -2  \\
            0 & 3  & 8  & 5  & -7  \\
            0 & 0  & 0  & 5  & -5  \\
            0 & 0  & 14 & -4 & -12 \\
            0 & 0  & 0  & 5  & 11
          \end{vmatrix}
          \\
          R_3 \leftrightarrow R_4 \label{eq:1.1.2}
          &\quad
          \begin{vmatrix}
            1 & -2 & -1 & 0  & -2  \\
            0 & 3  & 8  & 5  & -7  \\
            0 & 0  & 14 & -4 & -12 \\
            0 & 0  & 0  & 5  & -5  \\
            0 & 0  & 0  & 5  & 11
          \end{vmatrix}
          \\
          R_5 - R_4 \rightarrow R_5
          &\quad
          \begin{vmatrix}
            1 & -2 & -1 & 0  & -2  \\
            0 & 3  & 8  & 5  & -7  \\
            0 & 0  & 14 & -4 & -12 \\
            0 & 0  & 0  & 5  & -5  \\
            0 & 0  & 0  & 0  & 16
          \end{vmatrix}
        .\end{align}
        \begin{center}
          Because of the row exchange in \eqref{eq:1.1.2}, the sign of the determinant is changed.
        \end{center}
        \begin{align}
          \Delta &= -( 1 \times 3 \times 14 \times 5 \times 16 ) \\
                 &= -3360
        .\end{align}
    \end{enumerate}
  \item Let $U$ be a square matrix such that $U^\intercal U=I$. Show that $\det(U)=\pm1$. 
      \begin{align}
        \det(U^\intercal U) &= \det(I) \\
        \det(U^\intercal) \det(U) &= 1 \\
        \det(U) \det(U) &= 1 \\
        \det(U)^2 &= 1 \\
        \det(U) &= \pm 1
      .\end{align}
\end{enumerate}

\newpage

%%%%%%%%%%%%%%%%%%%%
\section*{Question (2)}
%%%%%%%%%%%%%%%%%%%%

For
$A =
  \begin{bmatrix}
    1 & 0 & 0 & 0 \\
    1 & 1 & 0 & 0 \\
    0 & 1 & 1 & 0 \\
    0 & 0 & 1 & 1
  \end{bmatrix},
  \quad
  b = \begin{bmatrix} 1\\2\\3\\4 \end{bmatrix}$ Find

\begin{enumerate}[label=\roman*.]
  \item $A^{-1}$.
    \begin{align}
      \begin{bmatrix} 
        \begin{array}{cccc|cccc}
          1 & 0 & 0 & 0 & 1 & 0 & 0 & 0 \\
          1 & 1 & 0 & 0 & 0 & 1 & 0 & 0 \\
          0 & 1 & 1 & 0 & 0 & 0 & 1 & 0 \\
          0 & 0 & 1 & 1 & 0 & 0 & 0 & 1
        \end{array}
      \end{bmatrix} 
    .\end{align}
    \begin{align}
      R_2 - R_1 \rightarrow R_2
      &\quad
      \begin{bmatrix} 
        \begin{array}{cccc|cccc}
          1 & 0 & 0 & 0 & 1  & 0 & 0 & 0 \\
          0 & 1 & 0 & 0 & -1 & 1 & 0 & 0 \\
          0 & 1 & 1 & 0 & 0  & 0 & 1 & 0 \\
          0 & 0 & 1 & 1 & 0  & 0 & 0 & 1
        \end{array}
      \end{bmatrix} \\
      R_3 - R_2 \rightarrow R_3
      &\quad
      \begin{bmatrix} 
        \begin{array}{cccc|cccc}
          1 & 0 & 0 & 0 & 1  & 0  & 0 & 0 \\
          0 & 1 & 0 & 0 & -1 & 1  & 0 & 0 \\
          0 & 0 & 1 & 0 & 1  & -1 & 1 & 0 \\
          0 & 0 & 1 & 1 & 0  & 0  & 0 & 1
        \end{array}
      \end{bmatrix} \\
      R_4 - R_3 \rightarrow R_4
      &\quad
      \begin{bmatrix} 
        \begin{array}{cccc|cccc}
          1 & 0 & 0 & 0 & 1  & 0  & 0  & 0 \\
          0 & 1 & 0 & 0 & -1 & 1  & 0  & 0 \\
          0 & 0 & 1 & 0 & 1  & -1 & 1  & 0 \\
          0 & 0 & 0 & 1 & -1 & 1  & -1 & 1
        \end{array}
      \end{bmatrix}
    .\end{align}
    \begin{align}
      A^{-1} &=
      \begin{bmatrix} 
        1  & 0  & 0  & 0 \\
        -1 & 1  & 0  & 0 \\
        1  & -1 & 1  & 0 \\
        -1 & 1  & -1 & 1
      \end{bmatrix} 
    .\end{align}
  \item $\left|2A^2\right|$
    \begin{align}
      \left|2A^2\right| &= 2^4 \left|A^2\right| \\
                        &= 16 \left|A^2\right| \\
                        &= 16 \left|A\right|^2 \\
                        &= 16 \left( 1 \right)^2 \\
                        &= 16
    .\end{align}
  \item $x$ such that $A^\intercal x=b$
    \begin{align}
      \begin{bmatrix} 
        \begin{array}{cccc|c}
          1 & 1 & 0 & 0 & 1 \\
          0 & 1 & 1 & 0 & 2 \\
          0 & 0 & 1 & 1 & 3 \\
          0 & 0 & 0 & 1 & 4
        \end{array}
      \end{bmatrix}
    .\end{align}
    \begin{align}
      R_1 - R_2 \rightarrow R_1
      &\quad
      \begin{bmatrix} 
        \begin{array}{cccc|c}
          1 & 0 & -1 & 0 & -1 \\
          0 & 1 & 1  & 0 & 2  \\
          0 & 0 & 1  & 1 & 3  \\
          0 & 0 & 0  & 1 & 4
        \end{array}
      \end{bmatrix} \\
      R_2 - R_3 \rightarrow R_2
      &\quad
      \begin{bmatrix} 
        \begin{array}{cccc|c}
          1 & 0 & -1 & 0 & -1 \\
          0 & 1 & 0  & -1& -1 \\
          0 & 0 & 1  & 1 & 3  \\
          0 & 0 & 0  & 1 & 4
        \end{array}
      \end{bmatrix} \\
      R_1 + R_3 \rightarrow R_1
      &\quad
      \begin{bmatrix} 
        \begin{array}{cccc|c}
          1 & 0 & 0 & 1 & 2 \\
          0 & 1 & 0 & -1& -1 \\
          0 & 0 & 1 & 1 & 3  \\
          0 & 0 & 0 & 1 & 4
        \end{array}
      \end{bmatrix} \\
      R_1 - R_4 \rightarrow R_1
      &\quad
      \begin{bmatrix} 
        \begin{array}{cccc|c}
          1 & 0 & 0 & 0 & -2 \\
          0 & 1 & 0 & -1& -1 \\
          0 & 0 & 1 & 1 & 3  \\
          0 & 0 & 0 & 1 & 4
        \end{array}
      \end{bmatrix} \\
      R_2 + R_4 \rightarrow R_2
      &\quad
      \begin{bmatrix} 
        \begin{array}{cccc|c}
          1 & 0 & 0 & 0 & -2 \\
          0 & 1 & 0 & 0 & 3  \\
          0 & 0 & 1 & 1 & 3  \\
          0 & 0 & 0 & 1 & 4
        \end{array}
      \end{bmatrix} \\
      R_3 - R_4 \rightarrow R_3
      &\quad
      \begin{bmatrix} 
        \begin{array}{cccc|c}
          1 & 0 & 0 & 0 & -2 \\
          0 & 1 & 0 & 0 & 3  \\
          0 & 0 & 1 & 0 & -1 \\
          0 & 0 & 0 & 1 & 4
        \end{array}
      \end{bmatrix}
    .\end{align}
    \begin{align}
      x &=
      \begin{bmatrix} 
        -2 \\ 3 \\ -1 \\ 4
      \end{bmatrix}
    .\end{align}
\end{enumerate}

\newpage

%%%%%%%%%%%%%%%%%%%%
\section*{Question (3)}
%%%%%%%%%%%%%%%%%%%%

If $C=\begin{bmatrix} 1&0&0&0\\0&1&0&0\\0&a&1&0\\0&0&0&1 \end{bmatrix}, a \in \mathbb{R}$
\begin{enumerate}[label=\roman*.]
  \item Find $Tr\left( C+2C^\intercal \right)$.
    \begin{align}
      C+2C^\intercal &=
      \begin{bmatrix} 
        1 & 0 & 0 & 0 \\
        0 & 1 & 0 & 0 \\
        0 & a & 1 & 0 \\
        0 & 0 & 0 & 1
      \end{bmatrix} +
      2\begin{bmatrix} 
        1 & 0 & 0 & 0 \\
        0 & 1 & a & 0 \\
        0 & 0 & 1 & 0 \\
        0 & 0 & 0 & 1
      \end{bmatrix} \\
      &=
      \begin{bmatrix} 
        3 & 0 & 0  & 0 \\
        0 & 3 & 2a & 0 \\
        0 & a & 3  & 0 \\
        0 & 0 & 0  & 3
      \end{bmatrix}\\
      Tr\left( C+2C^\intercal \right) &= 3+3+3+3 \\
                                      &= 12
    .\end{align}
  \item Find, if possible $C^{-1}$.
    \begin{align}
      \begin{bmatrix} 
        \begin{array}{cccc|cccc}
          1 & 0 & 0 & 0 & 1 & 0 & 0 & 0 \\
          0 & 1 & 0 & 0 & 0 & 1 & 0 & 0 \\
          0 & a & 1 & 0 & 0 & 0 & 1 & 0 \\
          0 & 0 & 0 & 1 & 0 & 0 & 0 & 1
        \end{array}
      \end{bmatrix}
    .\end{align}
    \begin{align}
      R_3 - aR_2 \rightarrow R_3
      &\quad
      \begin{bmatrix} 
        \begin{array}{cccc|cccc}
          1 & 0 & 0 & 0 & 1 & 0 & 0 & 0 \\
          0 & 1 & 0 & 0 & 0 & 1 & 0 & 0 \\
          0 & 0 & 1 & 0 & 0 & -a & 1 & 0 \\
          0 & 0 & 0 & 1 & 0 & 0 & 0 & 1
        \end{array}
      \end{bmatrix} \\
    .\end{align}
    \begin{align}
      C^{-1} &=
      \begin{bmatrix} 
        1 & 0 & 0 & 0 \\
        0 & 1 & 0 & 0 \\
        0 & -a & 1 & 0 \\
        0 & 0 & 0 & 1
      \end{bmatrix}
    .\end{align}
  \item Find, if possible $\left|2C^{-1}\right|$, $\left|\left( 2C \right)^{-1}\right|$.
    \begin{align}
      \left| 2C^{-1} \right| &= 2^4\left| C^{-1} \right| \\
                             &= 16\left| C^{-1} \right| \\
                             &= 16\left( 1 \right) \\
                             &= 16
    .\end{align}
    \begin{align}
      \left| \left( 2C \right)^{-1} \right| &= \frac{1}{2^4}\left| C^{-1} \right| \\
                                            &= \frac{1}{16}\left| C^{-1} \right| \\
                                            &= \frac{1}{16}\left( 1 \right) \\
                                            &= \frac{1}{16}
    .\end{align}
  \item Derive a formula for $C^n$.
    \begin{align}
      C^n &=
      \begin{bmatrix} 
        1 & 0 & 0 & 0 \\
        0 & 1 & 0 & 0 \\
        0 & na & 1 & 0 \\
        0 & 0 & 0 & 1
      \end{bmatrix}
    .\end{align}
\end{enumerate}

\newpage

%%%%%%%%%%%%%%%%%%%%
\section*{Question (4)}
%%%%%%%%%%%%%%%%%%%%

Mark each statement True or False. Justify each answer.

\begin{enumerate}[label=\roman*.]
  \item If $A$ can be row reduced to the identity matrix, then $A$ must be
    invertible. 
    \begin{center}
      \textbf{True}; since $A$ can be row reduced to the identity matrix, $A$
      is row equivalent to the identity matrix and every matrix that is row
      equivalent to the identity is invertible. 
    \qedsymbol\end{center}
  \item If $A$ is invertible, then elementary row operations that reduce $A$ to
    the identity In also reduce $A^{-1}$ to $I_n$. 
    \begin{center}
      \textbf{False}; elementary row operations that reduce $A$ to the identity
      $I_n$ does not necessarily reduce $A^{-1}$ to $I_n$. 
    \qedsymbol\end{center}
  \item The columns of an $n\times n$ matrix $A$ span $\mathbb{R}^n$ when $A$ is
    invertible. 
    \begin{center}
      \textbf{True}; the columns of an $n\times n$ matrix $A$ span
      $\mathbb{R}^n$ when $A$ is invertible as the columns of $A$ are
      linearly independent. 
    \qedsymbol\end{center}
  \item If the $n\times n$ matrices $E$ and $F$ have the property that $EF=I$, then
    $E$ and $F$ commute. 
    \begin{center}
      \textbf{True}; if $EF=I$, then $E=F^{-1}$ and $F=E^{-1}$ so $EF=FE=I$.
    \qedsymbol\end{center}
  \item The determinant of $A_n$ is the product of the pivots in any echelon
    form $U$ of $A_n$, multiplied by $(1)^r$, where $r$ is the number of row
    interchanges made during row reduction from $A_n$ to $U$. 
    \begin{center}
      \textbf{False}; $(-1)^r$ not $(1)^r$.
    \qedsymbol\end{center}
  \item $|-A|$ = $-|A|$. 
    \begin{center}
      \textbf{False}; $|-A|$ = $(-1)^n|A|$ and $n$ could be even.
    \qedsymbol\end{center}
  \item If $A_{n\times n}$ is reduced to an upper triangular matrix $U$ through row replacement operations only, then $|A|$ = $|U|$. 
    \begin{center}
      \textbf{False}; $|A|$ = $(-1)^r|U|$.
    \qedsymbol\end{center}
  \item If $A_{n\times n}$ is skew-symmetric and $n$ is an odd positive integer,
    then $A$ is non-invertible. 
    \begin{center}
      \textbf{True}; if $A^T=-A$ and $n$ is an odd positive integer, then $|A|$ =
      $|A^T|$ = $|-A|$ = $-|A|$ = $0$ so $A$ is non-invertible.
    \qedsymbol\end{center}
  \item For any square matrices $A$, $B$, we have $Tr(BAB)=Tr(AB^2)$.  
    \begin{center}
      \textbf{True}; because of the cyclic property of trace.
    \qedsymbol\end{center}
\end{enumerate}

\end{document}
