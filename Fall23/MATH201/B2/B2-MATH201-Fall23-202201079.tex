\documentclass{zc-ust-hw}

\usepackage{thmtools}

% Theorem
\theoremstyle{definition}
\newtheorem{theorem}{Theorem}

\name{SalahDin Ahmed Salh Rezk}
\id{202201079}
\course{Linear Algebra (MATH 201)}
\assignment{Bonus 2}

\begin{document}

\maketitle

\begin{enumerate}

  %%%%%%%%%%%%%%%%%%%%
  \item Prove that every square matrix can be written as a matrix
    multiplication of two symmetric matrices
  %%%%%%%%%%%%%%%%%%%%

    \begin{proof}
      Assume simpler case of $2 \times 2$ matrix:
      \begin{equation}
        A=\begin{bmatrix} a_{11} & a_{12}\\ a_{21} & a_{22} \end{bmatrix}
        \quad
        B=\begin{bmatrix} b_{11} & b_{12} \\ b_{12} & b_{22} \end{bmatrix}
        \quad
        C=\begin{bmatrix} c_{11} & c_{12} \\ c_{12} & c_{22} \end{bmatrix}
        \quad
        \text{such that}
        \quad
        A=BC
      .\end{equation}
      \begin{equation}
        BC =
        \begin{bmatrix} 
          b_{11}\cdot c_{11} + b_{12}\cdot c_{12} & b_{11}\cdot c_{12} + b_{12}\cdot c_{22} \\
          b_{21}\cdot c_{11} + b_{22}\cdot c_{12} & b_{21}\cdot c_{12} + b_{22}\cdot c_{22}
        \end{bmatrix} 
      .\end{equation}
      \begin{equation}
        \implies \begin{cases}
          a_{11} = b_{11}\cdot c_{11} + b_{12}\cdot c_{12} \\
          a_{12} = b_{11}\cdot c_{12} + b_{12}\cdot c_{22} \\
          a_{21} = b_{21}\cdot c_{11} + b_{22}\cdot c_{12} \\
          a_{22} = b_{21}\cdot c_{12} + b_{22}\cdot c_{22}
        \end{cases}
      \end{equation}
      In this system of equations, we have 6 unknowns and 4 equations, thus we
      can choose any two variables to be free variables. Let's choose $b_{11}$
      and $b_{22}$ to be free variables. Then we can solve the system of
      equations for $b_{12}$ and $b_{21}$:
      \begin{equation}
        \begin{cases}
          b_{12} = \dfrac{a_{12} - b_{11}\cdot c_{12}}{c_{22}} \\
          \\
          b_{21} = \dfrac{a_{21} - b_{22}\cdot c_{12}}{c_{11}}
        \end{cases}
      \end{equation}
      Thus, there always exists a solution for $B$ and $C$ such that $A=BC$, where
      $B$ and $C$ are symmetric matrices, such that $c_{22}, c_{11} \neq 0$.

      This can be generalized to any $n \times n$ matrix as follows:
      \begin{equation}
        \sum_{i=1}^{n} \sum_{j=1}^{n} a_{ij} = \sum_{i=1}^{n} \sum_{j=1}^{n} b_{ij} \cdot c_{ij}
        \iff
        \begin{cases}
          a_{11} = b_{11}\cdot c_{11} + b_{12}\cdot c_{12} + \cdots + b_{1n}\cdot c_{1n} \\
          a_{12} = b_{11}\cdot c_{12} + b_{12}\cdot c_{22} + \cdots + b_{1n}\cdot c_{2n} \\
          \vdots \\
          a_{1n} = b_{11}\cdot c_{1n} + b_{12}\cdot c_{2n} + \cdots + b_{1n}\cdot c_{nn} \\
          a_{21} = b_{21}\cdot c_{11} + b_{22}\cdot c_{12} + \cdots + b_{2n}\cdot c_{1n} \\
          \vdots \\
          a_{nn} = b_{n1}\cdot c_{1n} + b_{n2}\cdot c_{2n} + \cdots + b_{nn}\cdot c_{nn}
        \end{cases}
      .\end{equation}
      \begin{equation}
        \implies \begin{cases}
          b_{12} = \dfrac{a_{12} - b_{11}\cdot c_{12} - \cdots - b_{1n}\cdot c_{2n}}{c_{22}} \\
          b_{21} = \dfrac{a_{21} - b_{22}\cdot c_{12} - \cdots - b_{2n}\cdot c_{1n}}{c_{11}} \\
          \vdots \\
          b_{1n} = \dfrac{a_{1n} - b_{11}\cdot c_{1n} - \cdots - b_{1,n-1}\cdot c_{nn}}{c_{nn}} \\
          b_{n1} = \dfrac{a_{n1} - b_{n2}\cdot c_{11} - \cdots - b_{n,n-1}\cdot c_{1n}}{c_{nn}} \\
        \end{cases}
      \end{equation}
      Therefore, there always exists $n^2$ equations and $n(n+1)$ unknowns,
      thus we can choose $n$ variables to be free variables. Then we can always
      solve the system of equations for the remaining $n^2 - n$ variables, such
      that $c_{ii} \neq 0$ for all $i \in \{1, 2, \cdots, n\}$.
    \end{proof}

  %%%%%%%%%%%%%%%%%%%%
  \item Given that A, B, and C are Three matrices...A commutes with B....B
    commutes with C, then prove that A commutes with some Polynomial matrix
    function of C.
  %%%%%%%%%%%%%%%%%%%%
    \begin{theorem}
      \label{thm:commutativity}
      If the set of matrices considered is restricted to Hermitian matrices
      without multiple eigenvalues, then commutativity is transitive.
      \begin{equation*}
        A\cdot B=B\cdot A \land B\cdot C=C\cdot B \implies A\cdot C=C\cdot A
      .\end{equation*}
    \end{theorem}
    Under this restriction, the proof is trivial:
    \begin{proof}
      Let $P(C)$ be a polynomial matrix function of $C$, such that
      \begin{equation}
        P(C) = \sum_{i=0}^{n} a_i C^i
      .\end{equation}
      As such, we are trying to prove that
      \begin{equation}
        A\cdot B=B\cdot A \land B\cdot C=C\cdot B \implies A\cdot P(C)=P(C)\cdot A
      \end{equation}
      Then:
      \begin{align}
        A\cdot P(C) &\overset{?}{=} P(C)\cdot A \\
        A\cdot \sum_{i=0}^{n} a_i C^i &\overset{?}{=} \sum_{i=0}^{n} a_i C^i \cdot A \\
        A(a_0 I + a_1 C + a_2 C^2 + \cdots + a_n C^n) &\overset{?}{=} (a_0 I + a_1 C + a_2 C^2 + \cdots + a_n C^n) \cdot A \\
        a_0 A + a_1 AC + a_2 AC^2 + \cdots + a_n AC^n &\overset{?}{=} a_0 A + a_1 CA + a_2 C^2A + \cdots + a_n C^nA \\
        a_1 AC + a_2 AC^2 + \cdots + a_n AC^n &\overset{?}{=} a_1 CA + a_2 C^2A + \cdots + a_n C^nA
      \end{align}
      \begin{equation}
        \implies 
        \begin{cases}
          \begin{split}
            a_1 AC &\overset{?}{=} a_1 CA \\
            a_2 AC^2 &\overset{?}{=} a_2 C^2A \\
            \vdots \\
            a_n AC^n &\overset{?}{=} a_n C^nA
          \end{split}
        \end{cases}
        \implies
        \begin{cases}
          \begin{split}
            AC &\overset{?}{=} CA \\
            AC^2 &\overset{?}{=} C^2A \\
            \vdots \\
            AC^n &\overset{?}{=} C^nA
          \end{split}
        \end{cases}
      \end{equation}
      Using Theorem~\ref{thm:commutativity}, $AC^i = C^iA$ for all $i \in \N$, by induction.

      Therefore, $A\cdot P(C)=P(C)\cdot A$.
    \end{proof}
    Note that this proof does not hold for non-Hermitian matrices or Hermitian
    matrices with multiple eigenvalues, as Theorem~\ref{thm:commutativity} does
    not hold in those cases.

\end{enumerate}

\end{document}
