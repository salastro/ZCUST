\documentclass{zc-ust-hw}

\name{SalahDin Ahmed Salh Rezk}
\id{202201079}
\course{Linear Algebra (MATH 201)}
\assignment{Assignment 3}

\begin{document}

\maketitle

\begin{enumerate}
  \item Question 1
    \begin{enumerate}
      \item Let $Q$ be a reflection transformation about the line $Y=mX$, such that $m=\tan\theta$. Find $Q$, and show that $Q$ is a matrix transformation such that $Q\begin{pmatrix} x\\y\\1 \end{pmatrix}=\begin{pmatrix} x'\\y'\\1 \end{pmatrix} $, where $\begin{pmatrix} x\\y\\1 \end{pmatrix} $ is the homogeneous coordinates of the point $\begin{pmatrix} x\\y \end{pmatrix} $ and $\begin{pmatrix} x'\\y'\\1 \end{pmatrix} $ is the homogeneous coordinates of the reflected vector $\begin{pmatrix} x'\\y' \end{pmatrix} $.
        \begin{figure}[ht]
          \centering\fontsize{0.75em}{1em}
          \incfig{q-transformation-plot}
          \caption{}
          \label{fig:q-transformation-plot}
        \end{figure}
        \begin{sol}\,
          \begin{enumerate}[label=\arabic*)]
            \item Move the system to the origin.
            \item Rotate the system by $-\theta$.
            \item Reflect the system about the $X$-axis.
            \item Reverse the rotation and the translation.
          \end{enumerate}
          \begin{align}
            Q_1 &= \begin{bmatrix} 
              1 & 0 & 0\\
              0 & 1 & -c\\
              0 & 0 & 1
            \end{bmatrix} \\
            Q_2 &= \begin{bmatrix} 
              \cos\theta & \sin\theta & 0\\
              -\sin\theta & \cos\theta & 0\\
              0 & 0 & 1
            \end{bmatrix} : \theta=\tan^{-1}m \\
            Q_3 &= \begin{bmatrix} 
              1 & 0 & 0 \\
              0 & -1 & 0 \\
              0 & 0 & 1
            \end{bmatrix}
          .\end{align}
          \begin{align}
            Q_4 &= \begin{bmatrix} 
              \cos\theta & -\sin\theta & 0\\
              \sin\theta & \cos\theta & 0\\
              0 & 0 & 1
            \end{bmatrix} \\
            Q_5 &= \begin{bmatrix} 
              1 & 0 & 0\\
              0 & 1 & c\\
              0 & 0 & 1
            \end{bmatrix}
          .\end{align}
          \begin{align}
            Q &= Q_5Q_4Q_3Q_2Q_1 \\
              &= Q_5Q_4Q_3\begin{bmatrix} 
                \cos(\theta) & \sin(\theta) & -c \sin(\theta) \\
                -\sin(\theta) & \cos(\theta) & -c \cos(\theta) \\
                0 & 0 & 1
              \end{bmatrix}  \\
              &= Q_5Q_4\begin{bmatrix} 
                \cos(\theta) & \sin(\theta) & -c \sin(\theta) \\
                \sin(\theta) & -\cos(\theta) & c \cos(\theta) \\
                0 & 0 & 1
              \end{bmatrix}  \\
              &= Q_5\begin{bmatrix} 
                \cos^2(\theta) - \sin^2(\theta) & 2\cos(\theta)\sin(\theta) & -2c\cos(\theta)\sin(\theta) \\
                2\cos(\theta)\sin(\theta) & -\cos^2(\theta) + \sin^2(\theta) & -c(-\cos^2(\theta) + \sin^2(\theta)) \\
                0 & 0 & 1
              \end{bmatrix}  \\
              &= \begin{bmatrix} 
                \cos^2(\theta) - \sin^2(\theta) & 2\cos(\theta)\sin(\theta) & -2c\cos(\theta)\sin(\theta) \\
                2\cos(\theta)\sin(\theta) & -\cos^2(\theta) + \sin^2(\theta) & c - c(-\cos^2(\theta) + \sin^2(\theta)) \\
                0 & 0 & 1
              \end{bmatrix} \\
              &= \begin{bmatrix} 
                \cos2\theta & \sin2\theta & -c\sin2\theta \\
                \sin2\theta & -\cos(2x) & 2c \\
                0 & 0 & 1
              \end{bmatrix} \\
              &\text{Substitute } \theta = \tan^{-1}m \nonumber\\
              &= \begin{bmatrix} 
                \cos(2\arctan m) & \sin(2\arctan m) & -c\sin(2\arctan m) \\
                \sin(2\arctan m) & -\cos(2\arctan m) & 2c \\
                0 & 0 & 1
              \end{bmatrix} \\
              &\text{Using } \cos(2\arctan m) = \frac{1-m^2}{1+m^2} \text{ and } \sin(2\arctan m) = \frac{2m}{1+m^2} \nonumber\\
              Q &= \begin{bmatrix} 
                \frac{1-m^2}{1+m^2} & \frac{2m}{1+m^2} & -c\frac{2m}{1+m^2} \\
                \frac{2m}{1+m^2} & -\frac{1-m^2}{1+m^2} & 2c \\
                0 & 0 & 1
              \end{bmatrix}
            .\end{align}
        \end{sol}
      \item Reflect the given triangle with (2,6), (3,6), and (2.5,8) about the line $Y=\frac{1}{2}x+3$.
        \begin{sol}
          \begin{equation}
            m = \frac{1}{2} \implies \theta = \tan^{-1}\frac{1}{2} \quad c=3
          .\end{equation}
          \begin{equation}
            P_1 = \begin{pmatrix} 2\\6\\1 \end{pmatrix} \quad P_2 = \begin{pmatrix} 3\\6\\1 \end{pmatrix} \quad P_3 = \begin{pmatrix} 2.5\\8\\1 \end{pmatrix}
          .\end{equation}
          \begin{align}
              Q &= \begin{bmatrix} 
                \frac{1-m^2}{1+m^2} & \frac{2m}{1+m^2} & -c\frac{2m}{1+m^2} \\
                \frac{2m}{1+m^2} & -\frac{2m}{1+m^2} & 2c \\
                0 & 0 & 1
              \end{bmatrix} \\
            &= \begin{bmatrix} 
                \frac{3}{5} & \frac{4}{5} & -\frac{12}{5} \\
                \frac{4}{5} & -\frac{4}{5} & 6 \\
                0 & 0 & 1
              \end{bmatrix}
          .\end{align}
          \begin{align}
            P_1' &= QP_1 \\
            &= \begin{bmatrix} 
                \frac{3}{5} & \frac{4}{5} & -\frac{12}{5} \\
                \frac{4}{5} & -\frac{4}{5} & 6 \\
                0 & 0 & 1
              \end{bmatrix} \begin{pmatrix} 2\\6\\1 \end{pmatrix} \\
              &= \begin{pmatrix}\frac{18}{5}\\\frac{14}{5}\\1\end{pmatrix} \\
            P_2' &= QP_2 \\
            &= \begin{bmatrix} 
                \frac{3}{5} & \frac{4}{5} & -\frac{12}{5} \\
                \frac{4}{5} & -\frac{4}{5} & 6 \\
                0 & 0 & 1
              \end{bmatrix} \begin{pmatrix} 3\\6\\1 \end{pmatrix} \\
              &= \begin{pmatrix} \frac{21}{5}\\\frac{18}{5}\\1 \end{pmatrix} \\
            P_3' &= QP_3 \\
            &= \begin{bmatrix} 
                \frac{3}{5} & \frac{4}{5} & -\frac{12}{5} \\
                \frac{4}{5} & -\frac{4}{5} & 6 \\
                0 & 0 & 1
              \end{bmatrix} \begin{pmatrix} 2.5\\8\\1 \end{pmatrix} \\
              &= \begin{pmatrix} 5.5\\1.6\\1 \end{pmatrix}
          .\end{align}
          The reflected triangle coordinates are $(3.6,2.8)$, $(4.2,3.6)$, and $(5.5,1.6)$.
        \end{sol}
    \end{enumerate}

  \item Question 2
    \begin{enumerate}
      \item Find a basis for the subspace spanned by the given vectors. What is
        the dimension of the subspace? 
    \begin{equation*}
      \begin{bmatrix} 1\\-3\\2\\-4 \end{bmatrix},
      \begin{bmatrix} -3\\9\\-6\\12 \end{bmatrix},
      \begin{bmatrix} 2\\-1\\4\\2 \end{bmatrix},
      \begin{bmatrix} 4\\5\\-3\\7 \end{bmatrix}
    \end{equation*}
    \begin{sol}
      \begin{equation}
        V = \text{Col}\left(
        \begin{bmatrix} 
          1 & -3 & 2 & 4\\
          -3 & 9 & -1 & 5\\
          2 & -6 & 4 & -3\\
          -4 & 12 & 2 & 7
        \end{bmatrix} 
        \right)
      .\end{equation}
      \begin{align}
        R_2 \leftarrow R_2 + 3R_1
        &\quad
        \begin{bmatrix} 
          1 & -3 & 2 & 4\\
          0 & 0 & 5 & 17\\
          2 & -6 & 4 & -3\\
          -4 & 12 & 2 & 7
        \end{bmatrix} \\
        R_3 \leftarrow R_3 - 2R_1
        &\quad
        \begin{bmatrix} 
          1 & -3 & 2 & 4\\
          0 & 0 & 5 & 17\\
          0 & 0 & 0 & -11\\
          -4 & 12 & 2 & 7
        \end{bmatrix} \\
        R_4 \leftarrow R_4 + 4R_1
        &\quad
        \begin{bmatrix} 
          1 & -3 & 2 & 4\\
          0 & 0 & 5 & 17\\
          0 & 0 & 0 & -11\\
          0 & 0 & 10 & 23
        \end{bmatrix} \\
        R_4 \leftarrow R_4 - 2R_3
        &\quad
        \begin{bmatrix} 
          1 & -3 & 2 & 4\\
          0 & 0 & 5 & 17\\
          0 & 0 & 0 & -11\\
          0 & 0 & 0 & -11
        \end{bmatrix} \\
        R_4 \leftarrow R_4 - R_3
        &\quad
        \begin{bmatrix} 
          1 & -3 & 2 & 4\\
          0 & 0 & 5 & 17\\
          0 & 0 & 0 & -11\\
          0 & 0 & 0 & 0
        \end{bmatrix} \\
        R_3 \leftarrow R_3 / -11
        &\quad
        \begin{bmatrix} 
          1 & -3 & 2 & 4\\
          0 & 0 & 5 & 17\\
          0 & 0 & 0 & 1\\
          0 & 0 & 0 & 0
        \end{bmatrix}
      \end{align}
      \begin{align}
        R_2 \leftarrow R_2 - 17R_3
        &\quad
        \begin{bmatrix} 
          1 & -3 & 2 & 4\\
          0 & 0 & 5 & 0\\
          0 & 0 & 0 & 1\\
          0 & 0 & 0 & 0
        \end{bmatrix} \\
        R_2 \leftarrow R_2 / 5
        &\quad
        \begin{bmatrix} 
          1 & -3 & 2 & 4\\
          0 & 0 & 1 & 0\\
          0 & 0 & 0 & 1\\
          0 & 0 & 0 & 0
        \end{bmatrix} \\
        R_1 \leftarrow R_1 - 4R_3
        &\quad
        \begin{bmatrix} 
          1 & -3 & 2 & 0\\
          0 & 0 & 1 & 0\\
          0 & 0 & 0 & 1\\
          0 & 0 & 0 & 0
        \end{bmatrix} \\
        R_1 \leftarrow R_1 - 2R_2
        &\quad
        \begin{bmatrix} 
          1 & -3 & 0 & 0\\
          0 & 0 & 1 & 0\\
          0 & 0 & 0 & 1\\
          0 & 0 & 0 & 0
        \end{bmatrix}
      .\end{align}
      \begin{equation}
        B_V = \left\{ \begin{bmatrix} 1\\-3\\2\\-4 \end{bmatrix}, \begin{bmatrix} -3\\9\\-6\\12 \end{bmatrix}, \begin{bmatrix} 2\\-1\\4\\2 \end{bmatrix} \right\}
      .\end{equation}
    \end{sol}

  \item Suppose a $4 \times 7$ matrix $A$ has 3 pivot columns. Is
    $\text{Col}(A)=\mathbb{R}^3$?  What is the dimension of $\text{Nul}(A)$?
    Explain your answers. 
    \begin{sol}
      False, $\text{Col}(A)$ does not equal $\mathbb{R}^3$, because the number
      of rows is 4, and thus it will span a 3-dimensional subspace in $\R^4$.
      The dimension of $\text{Nul}(A)$ is 4, because the dimension of the
      nullspace of a matrix is equal to the number of free variables in the
      reduced echelon form of the matrix, which is equal to number of columns
      minus number of pivots ($7-3=4$). Since there are 4 free variables in the
      reduced echelon form of $A$, the dimension of $\text{Nul}(A)$ is 4.
    \end{sol}

    \end{enumerate}

  \item Respond as comprehensively as possible and justify your answer.
    \begin{enumerate}
      \item Suppose $F$ is a $5\times 5$ matrix whose column space is not equal to
        $\R^{5}$. What can be said about $\text{Nul}(F)$? 
        \begin{sol}
          $\text{dim}(\text{Nul}(F)) > 0$ (the nullspace $Fx=0$ is nontrivial---the columns are linearly dependent).
        \end{sol}
      \item If $B$ is a $7\times 7$ matrix and $\text{Col}(B)=R^{7}$, what can
        be said about solutions of equations of the form
        $Bx=b$ for $b$ in $\R^{7}$?
        \begin{sol}
          $B$ has a solution for every $b$ in $\R^{7}$ (system is consistent).
        \end{sol}
      \item What can be said about the shape of  $m\times n$ matrix $A$
        when the columns of $A$ form a basis for $\R^{m}$?
        \begin{sol}
          $m \geq n$ (number of rows is greater than or equal to the number of
          columns--the matrix is "tall" or "full-rank" with m linearly independent columns).
        \end{sol}
    \end{enumerate}

  \item Question 4
    \begin{enumerate}
      \item Find the vector $x$ determined by the given coordinate vector $[x]_\beta$, and
        the given basis $\beta$, illustrate your answer graphically. 
        \begin{equation*}
          \beta=\left\{ \begin{bmatrix} 1\\1 \end{bmatrix}, \begin{bmatrix} 2\\-1 \end{bmatrix}  \right\}, [x]_\beta=\begin{bmatrix} 3\\2 \end{bmatrix} 
        \end{equation*}

        \begin{sol}
          \begin{align}
            \beta [x]_\beta  &= x\\
            \begin{bmatrix} 1 & 2\\1 & -1 \end{bmatrix} \begin{bmatrix} 3\\2 \end{bmatrix} &= \begin{bmatrix} 7\\1 \end{bmatrix}
            % \beta [x]_\beta  &= \\
            % \begin{bmatrix} 1 & 2\\1 & -1 \end{bmatrix} x &= \begin{bmatrix} 3\\2 \end{bmatrix} \\
            % x &= \begin{bmatrix} 1 & 2\\1 & -1 \end{bmatrix}^{-1} \begin{bmatrix} 3\\2 \end{bmatrix} \\
            %   &= \frac{1}{3} \begin{bmatrix} -1 & -2\\-1 & 1 \end{bmatrix} \begin{bmatrix} 3\\2 \end{bmatrix} \\
            %   &= \frac{1}{3} \begin{bmatrix} -7\\-1 \end{bmatrix} \\
            %   &= \begin{bmatrix} -\frac{7}{3}\\-\frac{1}{3} \end{bmatrix}
          .\end{align}
        \end{sol}

        \begin{figure}[H]
          \begin{center}
            \begin{tikzpicture}[>=stealth, scale=0.8, transform shape]
              % x and y axis
              \draw[<->] (-2,0) -- (7,0) node[right] {$x$};
              \draw[<->] (0,-2) -- (0,3) node[above] {$y$};

              % i and j hat
              \draw[->, thick, black] (0,0) -- (1,0) node[below] {$\hat{\imath}$};
              \draw[->, thick, black] (0,0) -- (0,1) node[left] {$\hat{\jmath}$};

              % beta basis
              \draw[->, thick, blue] (0,0) -- (1,1);
              \draw[->, thick, blue] (0,0) -- (2,-1);

              % x beta vector
              % \draw[->, thick, red] (0,0) -- (3,2) node[above] {$[x]_\beta=\begin{bmatrix} 3\\2 \end{bmatrix}$};

              % x vector
              \draw[->, thick, teal] (0,0) -- (7,1) node[above] {$\vec{x}=\begin{bmatrix} 7\\1 \end{bmatrix}$};

              % x beta vector components
              \draw[->, thick, blue] (1,1) -- (2,2);
              \draw[->, thick, blue] (1,1) -- (2,2);
              \draw[->, thick, blue] (2,2) -- (3,3) node[above]
                {$\beta_{\hat{\jmath}}=\begin{bmatrix} 1\\1 \end{bmatrix}$};
              \draw[->, thick, blue] (2,-1) -- (4,-2) node[below]
                {$\beta_{\hat{\imath}}=\begin{bmatrix} 2\\-1 \end{bmatrix}$};

              % sum of components
              \draw[->, thick, red] (3,3) -- (7,1);
              \draw[->, thick, red] (4,-2) -- (7,1);

            \end{tikzpicture}
          \end{center}
          \caption{}%
          \label{fig:}
        \end{figure}

      \item Find the $LU$ factorization of matrix $A=\begin{bmatrix} 4&-8&8&-4\\16&-29&27&-14\\-1&-10&18&-4 \end{bmatrix}$.
        \begin{sol}
          \begin{equation}
            \begin{bmatrix}
              4  & -8  & 8  & -4  \\
              16 & -29 & 27 & -14 \\
              -1 & -10 & 18 & -4
            \end{bmatrix}
          \end{equation}
          \begin{align}
            \xrightarrow{R_2 = R_2 - 4R_1}
            \quad
            &\begin{bmatrix}
              4  & -8  & 8  & -4 \\
              0  & 3   & -5 & 2  \\
              -1 & -10 & 18 & -4
            \end{bmatrix} \\
            \xrightarrow{R_3 = R_3 + \frac{1}{4}R_1}
            \quad
            &\begin{bmatrix}
              4 & -8  & 8  & -4 \\
              0 & 3   & -5 & 2  \\
              0 & -12 & 20 & -5
            \end{bmatrix} \\
            \xrightarrow{R_3 = R_3 + 4R_2}
            \quad
            &\begin{bmatrix}
              4 & -8 & 8  & -4 \\
              0 & 3  & -5 & 2  \\
              0 & 0  & 0  & 3
            \end{bmatrix}
          .\end{align}
          \begin{equation}
            U = \begin{bmatrix} 4 & -8 & 8 & -4 \\ 0 & 3 & -5 & 2 \\ 0 & 0 & 0 & 3 \end{bmatrix}
          \end{equation}
          \begin{gather}
            E_1 = \begin{bmatrix} 1 & 0 & 0 \\ -4 & 1 & 0 \\ 0 & 0 & 1 \end{bmatrix} \quad E_2 = \begin{bmatrix} 1 & 0 & 0 \\ 0 & 1 & 0 \\ \frac{1}{4} & 0 & 1 \end{bmatrix} \quad E_3 = \begin{bmatrix} 1 & 0 & 0 \\ 0 & 1 & 0 \\ 0 & 4 & 1 \end{bmatrix} \\
            L = E_1^{-1}E_2^{-1}E_3^{-1} = \begin{bmatrix} 1 & 0 & 0 \\ 4 & 1 & 0 \\ -\frac{1}{4} & -4 & 1 \end{bmatrix} \\
          .\end{gather}
        \end{sol}

    \end{enumerate}

  \item Given that,
    \begin{equation}
      A=\left(\begin{array}{ccc}
          3 & -6 & -4 \\
          -3 & 2 & 3 \\
          6 & 8 & -4
      \end{array}\right)
      \quad
      U=\left(\begin{array}{ccc}
          3 & -6 & -4 \\
          0 & -4 & -1 \\
          0 & 0 & -1
      \end{array}\right)
    \end{equation}
    If $A$ is reduced to the row echelon form $U$ using only row replacement operations,
    \begin{enumerate}
      \item Find $L$ such that $A=LU$.
        \begin{sol}
          \begin{gather}
            E_1 = \begin{bmatrix} 1 & 0 & 0 \\ 1 & 1 & 0 \\ 0 & 0 & 1 \end{bmatrix}
            \quad
            E_2 = \begin{bmatrix} 1 & 0 & 0 \\ 0 & 1 & 0 \\ -2 & 0 & 1 \end{bmatrix}
            \quad
            E_3 = \begin{bmatrix} 1 & 0 & 0 \\ 0 & 1 & 0 \\ 0 & 5 & 1 \end{bmatrix} \\
            L = E_1^{-1}E_2^{-1}E_3^{-1} = \begin{bmatrix} 1 & 0 & 0 \\ -1 & 1 & 0 \\ 2 & -5 & 1 \end{bmatrix}
          \end{gather}
        \end{sol}
      \item Find $|A|$.
        \begin{sol}
          \begin{align}
            |A| &= |LU| \\
            &= |L||U| \\
            &= 3 \cdot (-4) \cdot (-1) \\
            &= 12
          .\end{align}
        \end{sol}
      \item Find a basis for Col $A$.
        \begin{sol}
          \begin{align}
            \text{Col } A &= \text{Col } U \\
            &= \text{Col } \begin{bmatrix} 3 & -6 & -4 \\ 0 & -4 & -1 \\ 0 & 0 & -1 \end{bmatrix} \\
            &= \text{Span} \left\{ \begin{bmatrix} 3 \\ 0 \\ 0 \end{bmatrix}, \begin{bmatrix} -6 \\ -4 \\ 0 \end{bmatrix}, \begin{bmatrix} -4 \\ -1 \\ -1 \end{bmatrix} \right\}
          .\end{align}
        \end{sol}
      \item Find rank $A$.
        \begin{sol}
          \begin{align}
            \text{rank } A &= \text{rank } U \\
            &= 3
          .\end{align}
        \end{sol}
      \item Find Nul $A$.
        \begin{sol}
          \begin{align}
            \text{Nul } A &= \text{Nul } U \\
                          &= \begin{bmatrix} 0\\0\\0 \end{bmatrix} 
          .\end{align}
        \end{sol}
    \end{enumerate}

    \newpage

  \item Question 6
    \begin{enumerate}
      \item Let $M_{2\times 2}$ be the vector space of all $2\times 2$ matrices, and define
        \[
          T: M_{2\times 2} \to M_{2\times 2} \quad \text{by} \quad T(A) = A+A^T \quad \text{, where} \begin{bmatrix} a&b\\c&d \end{bmatrix} 
        .\] 
        \begin{enumerate}
          \item Show that $T$ is a linear transformation.
            \begin{sol}
              \begin{align}
                T(A+B) &= (A+B) + (A+B)^T \\
                       &= (A+B) + (A^T + B^T) \\
                       &= A + A^T + B + B^T \\
                       &= T(A) + T(B) \\
                T(cA) &= cA + (cA)^T \\
                      &= cA + cA^T \\
                      &= c(A + A^T) \\
                      &= cT(A)
              .\end{align}
            \end{sol}

          \item Let $B$ be any element of $M_{2\times 2}$ such that $B^\intercal=B$. Find a matrix  $A$ in $M_{2\times 2}$ such that $T(A)=B$.
            \begin{sol}
              \begin{align}
                T(A) &= B \\
                A + A^T &= B \\
                \begin{bmatrix} 
                  a_{11} & a_{12} \\
                  a_{21} & a_{22}
                \end{bmatrix} 
                +
                \begin{bmatrix} 
                  a_{11} & a_{21} \\
                  a_{12} & a_{22}
                \end{bmatrix}
                &=
                \begin{bmatrix} 
                  b_{11} & b_{12} \\
                  b_{12} & b_{22}
                \end{bmatrix} \\
                \begin{bmatrix} 
                  2a_{11} & a_{12} + a_{21} \\
                  a_{12} + a_{21} & 2a_{22}
                \end{bmatrix}
                &=
                \begin{bmatrix} 
                  b_{11} & b_{12} \\
                  b_{12} & b_{22}
                \end{bmatrix} \\
                \begin{bmatrix} 
                  a_{11} & a_{12} \\
                  a_{21} & a_{22}
                \end{bmatrix}
                &=
                \begin{bmatrix} 
                  \frac{b_{11}}{2} & a_{12} \\
                  b_{12} - a_{12} & \frac{b_{22}}{2}
                \end{bmatrix}
              .\end{align}
              Where $a_{12}$ is a free variable.
            \end{sol}
          \item Show that the range of $T$ is the set of B in $M_{2\times 2}$
            with the property that $B^\intercal=B$.
            \begin{sol}
              \begin{align}
                T(A)^\intercal &= (A+A^\intercal)^\intercal \\
                               &= A^\intercal + (A^\intercal)^\intercal \\
                               &= A^\intercal + A \\
                               &= A + A^\intercal \\
                               &= T(A)
              .\end{align}
              \begin{equation}
                \therefore \text{Range } T = \{B \in M_{2\times 2} \mid B^\intercal = B\}
              \end{equation}
            \end{sol}
            \newpage
          \item Describe the kernel of $T$.
            \begin{sol}
              \begin{align}
                T(A) &= 0 \\
                A + A^\intercal &= 0 \\
                A &= -A^\intercal \\
              .\end{align}
              The kernel of $T$ is the set of all skew-symmetric matrices.
            \end{sol}
        \end{enumerate}
      \item Consider the polynomials $P_1(t)=1+t^2$ and $P_2(t)=1-t^2$. Is
        $\{p_1,p_2\}$ a linearly independent set in $P_3$? Why or why not ? 
        \begin{sol}
          \begin{align}
            c_1P_1(t) + c_2P_2(t) &= 0 \\
            c_1(1+t^2) + c_2(1-t^2) &= 0 \\
            c_1 + c_2 + c_1t^2 - c_2t^2 &= 0 \\
            c_1 + c_2 + t^2(c_1 - c_2) &= 0
          \end{align}
          \begin{equation}
            \implies \begin{cases}
              c_1 + c_2 = 0 \\
              c_1 - c_2 = 0
            \end{cases}
            \implies \begin{cases}
              c_1 = 0 \\
              c_2 = 0
            \end{cases}
          .\end{equation}
          Therefore, $\{p_1,p_2\}$ is a linearly independent set in $P_3$.
        \end{sol}

      \item Let $H$ be the set of all vectors of the form $\begin{bmatrix} -2t\\5t\\3t \end{bmatrix} $. Find a vector $V$ in $\R^3$ such that $H=\text{Span}\{V\}$. Why does this show that $H$ is a subspace of $\R^3$? 
        \begin{sol}
          \begin{align}
            H &= \text{Span}\left\{ \begin{bmatrix} -2t\\5t\\3t \end{bmatrix} \right\} \\
              &= \left\{ \begin{bmatrix} -2t\\5t\\3t \end{bmatrix} \mid t \in \R \right\} \\
              &= \left\{ t\begin{bmatrix} -2\\5\\3 \end{bmatrix} \mid t \in \R \right\} \\
              &= \text{Span}\left\{ \begin{bmatrix} -2\\5\\3 \end{bmatrix} \right\}
          .\end{align}
          This shows that $H$ is a subspace of $\R^3$ because it is the span of a single vector.
        \end{sol}

    \end{enumerate}

\end{enumerate}

\end{document}
