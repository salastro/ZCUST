\documentclass[aspectratio=169, 10pt]{beamer}
\usetheme{metropolis}

\title{ReflectoRay}
\subtitle{Ray Reflection Simulation in Python}
\author{
Ahmed Mohammed Abdullah -- 202201293 \\
Salah Mahmoud Gamal -- 202201517 \\
SalahDin Ahmed Salh Rezk -- 202201079 \\
}
\date{\today}

\begin{document}

\maketitle

\section{Introduction}
\begin{frame}{Overview}
  \begin{itemize}
    \item The Python script simulates the reflection of rays off mirrors.
    \item Utilizes Turtle graphics for visualization.
    \item Options to save the simulation as an image or record it as a video.
  \end{itemize}
\end{frame}

\section{Script Components}
\begin{frame}{Script Components}
  \begin{itemize}
    \item \texttt{parse\_arguments}: Command line argument parsing.
    \item \texttt{load\_initial\_conditions}: Loading initial conditions from a JSON file.
    \item \texttt{setup\_screen}: Setting up the Turtle screen for simulation.
    \item \texttt{draw\_mirrors}: Drawing mirrors on the Turtle screen.
    \item \texttt{create\_ray}: Creating a Turtle object representing a ray.
    \item \texttt{simulate\_rays}: Simulating the reflection of rays off mirrors.
  \end{itemize}
\end{frame}

\section{Simulation Process}
\begin{frame}{Simulation Process}
  \begin{itemize}
    \item Mirrors, sources, and angles are initialized.
    \item Turtle graphics used for visualization.
    \item Rays are simulated, and reflections are calculated.
    \item Options to save images or record videos.
  \end{itemize}
\end{frame}

\section{Run Simulation}
\begin{frame}{Running the Simulation}
  \begin{itemize}
    \item Execute the script from the command line.
    \item Specify optional arguments: \texttt{-i} (image) and \texttt{-v} (video).
    \item Initial conditions loaded from \texttt{initial\_conditions.json}.
  \end{itemize}
\end{frame}

\section{Conclusion}
\begin{frame}{Conclusion}
  \begin{itemize}
    \item Turtle graphics provides a simple way to visualize ray reflection.
    \item Options for saving images and videos enhance the utility of the script.
    \item Further customization and improvements can be made based on specific requirements.
  \end{itemize}
\end{frame}

\end{document}

