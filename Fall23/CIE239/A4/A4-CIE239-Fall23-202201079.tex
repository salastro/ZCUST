\documentclass{zc-ust-hw}

\name{SalahDin Ahmed Salh Rezk}
\id{202201079}
\course{Digital Design and Computer Architecture (CIE 239)}
\assignment{Assignment 4}

\begin{document}

\maketitle

\begin{enumerate}

  \item Assume that registers $R_1$ and $R_{2}$ in the Figure hold two unsigned
    numbers. When select input $X$ is equal to 1, the adder–subtractor circuit
    performs the arithmetic operation ``$R_{1}$ + 2's complement of \(R_{2}\).''
    This sum and the output carry $C_n$ are transferred into $R_1$ and $C$ when
    $K_1 = 1$ and a positive edge occurs on the clock. 

    \begin{sol}\,
      \begin{itemize}
        \item Assume \( C=1 \).

      Perform \(R_{1}-R_{2}\) and $X=1$ 

      \( \therefore R_{1}=1111 \),\hspace{1em} suppose \( R_{2}=1010 \)
      2's complement of \( R_{2} \) is \( 0110 \)
      
      \( R_{1}-R_{2}=10101 \)
      
      \( \therefore R_{1}=0101 \) and \( C=1 \)

      \fbox{ So if \( C=1 \) then the value transferred to \( R_{1} \) is equal to \( R_{2}-R_{1} \) }

    \item Assume \( C=0 \).
      Perform \(R_{2}-R_{1}\)

      2's complement of \( R_{1} \) is \( 1010 \)

      \( R_{2}-R_{1}=1101 \)

      2's complement of \( R_{2}-R_{1} \) is \( 0011 \)

      \fbox{ So if \( C=0 \) then the 2's complement of \( R_{2}-R_{1} \) is equal to \(R_{1}\) }
  
      \end{itemize}
    \end{sol}

  \item The outputs of registers $R_0$, $R_1$, $R_2$, and $R_3$ are connected through
    4-to-1 multiplexers to the inputs of a fifth register, $R_4$. Each register is
    8 bits long. The required transfers, as dictated by four control variables,
    are 
    \begin{align*}
      C_{0}: R_{4} \leftarrow R_{0}, \\
      C_{1}: R_{4} \leftarrow R_{1}, \\
      C_{2}: R_{4} \leftarrow R_{2}, \\
      C_{3}: R_{4} \leftarrow R_{3}
    .\end{align*}
    
    \newpage
    
    The control variables are mutually exclusive (i.e., only one variable can
    be equal to 1 at any time) while the other three are equal to 0. Also, no
    transfer into $R_4$ is to occur for all control variables equal to 0.
    \begin{enumerate}
      \item Using registers and a multiplexer, draw a detailed logic diagram of
        the hardware that implements a single bit of these register transfers. 
        \begin{sol}\end{sol}
        \begin{figure}[H]
          \centering
          \begin{circuitikz}
            % Registers
            \draw (0,0) node[draw,minimum width=1cm,minimum height=1cm] (r0) {$R_0$};
            \draw (0,-2) node[draw,minimum width=1cm,minimum height=1cm] (r1) {$R_1$};
            \draw (0,-4) node[draw,minimum width=1cm,minimum height=1cm] (r2) {$R_2$};
            \draw (0,-6) node[draw,minimum width=1cm,minimum height=1cm] (r3) {$R_3$};
            % Mux
            \tikzset{mux 4by1/.style={muxdemux,muxdemux def={Lh=8, NL=4, Rh=4, NB=2, w=2, square pins=1}}}
            \node[mux 4by1] (mux) at (4,-4) {};
            % Connections
            \draw (r0.east) -- ++(1,0) |- (mux.lpin 1);
            \draw (r1.east) -- ++(1,0) |- (mux.lpin 2);
            \draw (r2.east) -- ++(1,0) |- (mux.lpin 3);
            \draw (r3.east) -- ++(1,0) |- (mux.lpin 4);
            \draw (mux.rpin 1) -- ++(1,0) node[right] {$R_4$};
            % Select labels
            \draw (mux.bpin 1) ++(-0.25,0) node[below] {\tt enbl};
            \draw (mux.bpin 2) ++(0.25,0) node[below] {\tt sel};
          \end{circuitikz}
          \caption{}
        \end{figure}
        
      \item Write HDL code for the design. 

        \begin{sol}\,
          \begin{lstlisting}[language=Verilog]
module Q_2 (clk, reset, C[0:3], R[0:4]);
  input clk, reset;
  input [3:0] C;
  input [7:0] R[0:3];
  output [7:0] R[4];
  reg [7:0] R[0:4];
  always @(posedge clk, posedge reset)
    if (reset)
      R[0:4] <= 0;
    else
      case (C)
        4'b0001: R[4] <= R[0];
        4'b0010: R[4] <= R[1];
        4'b0100: R[4] <= R[2];
        4'b1000: R[4] <= R[3];
        default: R[4] <= 0;
      endcase
endmodule
          \end{lstlisting}
        \end{sol}
    \end{enumerate}

  \item You are designing an FSM to keep track of the mood of four students
    working in the digital design lab. Each student’s mood is either HAPPY (the
    circuit works), SAD (the circuit blew up), BUSY (working on the circuit),
    CLUELESS (confused about the circuit), or ASLEEP (face down on the circuit
    board). How many states does the FSM have? What is the minimum number of
    bits necessary to represent these states? 
    \begin{sol}\,
      \begin{itemize}
        \item The FSM has \( 5^4=625 \) states.
        \item The minimum number of bits necessary to represent these states is \( \lceil \log_{2} 625 = \lceil 9.28 = 10 \) bits.
      \end{itemize}
    \end{sol}

  \item You have been enlisted to design a soda machine dispenser for your
    department lounge. Sodas are partially subsidized by the student chapter of
    the IEEE, so they cost only 25 cents. The machine accepts nickels, dimes,
    and quarters. When enough coins have been inserted, it dispenses the soda
    and returns any necessary change. Design an FSM controller for the soda
    machine. The FSM inputs are Nickel, Dime, and Quarter, indicating which
    coin was inserted. Assume that exactly one coin is inserted on each cycle.
    The outputs are Dispense, ReturnNickel, ReturnDime, and ReturnTwoDimes.
    When the FSM reaches 25 cents, it asserts Dispense and the necessary Return
    outputs required to deliver the appropriate change. Then it should be ready
    to start accepting coins for another soda. 

    \begin{sol}\end{sol}
    
    \begin{figure}[H]
      \centering
      \begin{tikzpicture}
        \tikzset{
          ->, % makes the edges directed
          node distance=5cm, % specifies the minimum distance between two nodes. Change if necessary.
          every state/.style={thick, fill=gray!10}, % sets the properties for each ’state’ node
          initial text=$ $, % sets the text that appears on the start arrow
        } 
        \node[state, initial, accepting] (0) {$s_{0}$};
        \node[state, above right of=0] (1) {$s_{1}$};
        \node[state, right of=1] (2) {$s_{2}$};
        \node[state, below right of=2] (3) {$s_{3}$};
        \node[state, below left of=3] (4) {$s_{4}$};
        \node[state, left of=4] (5) {$s_{5}$};
        \draw 
        (5) edge[loop above] node{n/D+rn} (5)
        (5) edge[loop below] node{d/D+rd} (5)
        (5) edge[loop left] node{q/D+rn+r2d} (5)
        (0) edge[above, bend left] node{n/0} (1)
        (0) edge[above, bend right] node{d/0} (2)
        (0) edge[above, bend right] node{q/D} (5)
        (1) edge[above, bend left] node{n/0} (2)
        (1) edge node{d/0} (3)
        (1) edge[above, bend right] node{q/D+rn+rd} (5)
        (2) edge[above, bend left] node{n/0} (3)
        (2) edge[above, bend right] node{d/0} (4)
        (2) edge[below] node{q/D+rd} (5)
        (3) edge[below, bend left] node{n/0} (4)
        (3) edge[above, bend left] node{d/D} (5)
        (3) edge node{q/D+rn+rd} (5)
        (4) edge[below, bend left] node{n/D} (5)
        (4) edge[below, bend right] node{d/D+rn} (5)
        (4) edge[below] node{q/D+r2d} (5);
          ;
      \end{tikzpicture}
      \caption{}
    \end{figure}
    
  \item Analyze the FSM shown in the figure. Write the state transition and
    output tables and sketch the state transition diagram. Describe in words
    what the FSM does.  Recall that the s and r register inputs indicate set
    and reset, respectively. 
    \begin{sol}
      \begin{equation}
        D_{0}=\overline{A}, \quad D_{1}=Q_{0}A, \quad D_{2}=(Q_{1}+Q_{2})A, \quad Q=Q_{2}
      \end{equation}
    \end{sol}
    \begin{table}[H]
      \centering
      \begin{tabular}{c|c|c|c|c|c|c|c|c|c|c}
        \( Q_{0} \) & \( Q_{1} \) & \( Q_{2} \) & \( A \) & \( Q_{0}' \) & \( Q_{1}' \) & \( Q_{2}' \) & \( D_{0} \) & \( D_{1} \) & \( D_{2} \) & \( Q \) \\
        \hline
        0 & 0 & 0 & 0 & 0 & 0 & 0 & 0 & 0 & 0 & 0 \\
        0 & 0 & 0 & 0 & 0 & 0 & 0 & 0 & 0 & 0 & 0 \\
        \end{tabular}
      \caption{}
    \end{table}
    
\end{enumerate}

\end{document}
