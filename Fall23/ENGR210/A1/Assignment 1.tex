\documentclass{zc-ust-hw}

\usepackage{lipsum}
\usepackage{circuitikz}
\usepackage{caption}
\usepackage{tabularx}
\usepackage{pgfplots}

\newcommand*{\name}{SalahDin Rezk}
\newcommand*{\id}{202201079}
\newcommand*{\course}{Electric Circuits (ENGR 210)}
\newcommand*{\assignment}{Assignment 1}

\begin{document}

\maketitle

\begin{enumerate}

  %%%%%%%%%%%%%%%%%%%%
  \item The numerical values of the voltages and currents in the
    interconnection seen in Fig. 1. Does the interconnection satisfy the power
    check?
    %%%%%%%%%%%%%%%%%%%%

    \begin{figure}[h]
      \centering
      \raisebox{-0.5\height}{
        \begin{circuitikz}[american, raised voltages]
          \draw (0,0) to [twoport, t=a, v^=$v_{a}$, f_=$i_a$] ++ (0,6);
          \draw (0,6) to (3,6);
          \draw (3,6) to [twoport, t=b, v_=$v_{b}$, f<=$i_a$] (3,3);
          \draw (3,3) to [twoport, t=c, v_=$v_{c}$, f=$i_c$] (3,0);
          \draw (3,0) to (0,0);
          \draw (3,6) to [twoport, t=d, v^<=$v_{d}$, f_<=$i_d$] (8,6);
          \draw (8,6) to [twoport, t=g, v^=$v_{g}$, f_<=$i_g$] (8,3);
          \draw (8,3) to [twoport, t=e, v_<=$v_{e}$, f^>=$i_e$] (3,3);
          \draw (8,3) to [twoport, t=h, v^<=$v_{h}$, f_<=$i_h$] (8,0);
          \draw (8,0) to [twoport, t=f, v^<=$v_{f}$, f_<=$i_f$] (3,0);
        \end{circuitikz}
      }
      \begin{tabularx}{\linewidth/3}{X X X}
        Element & Voltage & Current \\
        a & 990 & -22.5 \\
        b & 600 & -30 \\
        c & 300 & 60 \\
        d & 105 & 52.5 \\
        e & -120 & 30 \\
        f & 165 & 82.5 \\
        g & 585 & 52.5 \\
        h & -585 & 82.5 \\
      \end{tabularx}
      \caption{}
    \end{figure}

    \begin{align}
      P_{a} &= v_{a}i_{a} &&= 990 \times -22.5 &= -22275 \text{ W} \\
      P_{b} &= v_{b}i_{b} &&= 600 \times -30   &= -18000 \text{ W} \\
      P_{c} &= v_{c}i_{c} &&= 300 \times 60    &= 18000 \text{ W} \\
      P_{d} &= v_{d}i_{d} &&= 105 \times 52.5  &= 5512.5 \text{ W} \\
      P_{e} &= v_{e}i_{e} &&= -120 \times 30   &= -3600 \text{ W} \\
      P_{f} &= v_{f}i_{f} &&= 165 \times 82.5  &= 13612.5 \text{ W} \\
      P_{g} &= v_{g}i_{g} &&= 585 \times 52.5  &= 30712.5 \text{ W} \\
      P_{h} &= v_{h}i_{h} &&= -585 \times 82.5 &= -48262.5 \text{ W} \\
    .\end{align}
    \begin{align}
      P_{\text{total}} &= P_{a} + P_{b} + P_{c} + P_{d} + P_{e} + P_{f} + P_{g} + P_{h} \\
                       &= -22275 - 18000 + 18000 + 5512.5 - 3600 + 13612.5 + 30712.5 - 48262.5 \\
                       &= -24300
    .\end{align}

    \begin{center}
      \boxed{\text{The interconnection does not satisfy the power check.}}
    \end{center}

    \newpage

  \item The voltage and current at the terminals of the circuit element in Fig.
    2 are zero for $t < 0$. For $t \ge 0$ they are:
    \begin{align*}
      v &= 50e^{-1600t}-50e^{-400t} \text{ V} \\
      i &= 5e^{-1600t}-5e^{-400t} \text{ mA}
    \end{align*}
    \begin{enumerate}
      \item Find the power at $t=625 \,\mu\text{s}$.
        \begin{align}
          P(t) = v(t)i(t) =& (50e^{-1600t}-50e^{-400t})(5e^{-1600t}-5e^{-400t}) \times 10^{-3} \\
          P(625\times 10^{-6}) =&
                                \left( 
                                50e^{-1600(625\times 10^{-6})}-50e^{-400(625\times 10^{-6})}
                              \right)  \\
                                &\left( 
            5e^{-1600(625\times 10^{-6})}-5e^{-400(625\times 10^{-6})}
          \right) 
          \times 10^{-3} \\
                                &= 0.0420{ W}
        .\end{align}
        
      \item How much energy is delivered to the circuit element at $t=625\, \mu\text{s}$?
        \begin{align}
          E(t)&=\int_{0}^{t} P(x) \, dx \\
              &= \int_{0}^{625\times 10^{-6}} \left( 
                (50e^{-1600x}-50e^{-400x})(5e^{-1600x}-5e^{-400x}) \times 10^{-3}
              \right) \, dx \\
              &= 1.22 \times 10^{-5} \text{ J}
        .\end{align}
      \item Find the total energy delivered to the element.
        \begin{align}
          E_{\text{total}} &= \lim_{t\to\infty} E(t) \\
                           &= \lim_{t\to\infty} \int_{0}^{t} P(x) \, dx \\
                           &= \frac{9}{6400} \approx 1.41\times 10^{-4} \text{ J}
        .\end{align}
    \end{enumerate}

  \item An industrial battery is charged over a period of several hours at a
    constant voltage of 120 V. Initially, the current is 10 mA and increases
    linearly to 15 mA in 10 ks. From 10 ks to 20 ks, the current is constant at
    15 mA. From 20 ks to 30 ks the current decreases linearly to 10 mA. At 30
    ks the power is disconnected from the battery. 

    \begin{enumerate}
      \item Sketch the voltage, current and power from $t=0$ to $t=30$ ks.

        \begin{figure}[h]
          \centering
          \begin{tikzpicture}
            \begin{axis}[
              axis lines = left,
              xlabel = {$t$ ks},
              ylabel = {$i(t)$ mA},
              xmin = 0,
              xmax = 30,
              ymin = 0,
              ymax = 20,
              xtick = {0,10,20,30},
              xticklabels = {0,10,20,30},
              ytick = {0,5,10,15},
              yticklabels = {0,5, 10,15},
              ]
              \addplot [
                domain=0:10,
                samples=2,
                color=red,
                ]
                {0.5*x+10};
              \addplot [
                domain=10:20,
                samples=2,
                color=red,
                ]
                {15};
              \addplot [
                domain=20:30,
                samples=2,
                color=red,
                ]
                {-0.5*x+25};
            \end{axis}
          \end{tikzpicture}
          \caption{}
        \end{figure}
        
        \begin{figure}[h]
          \centering
          \begin{tikzpicture}
            \begin{axis}[
              axis lines = left,
              xlabel = {$t$ ks},
              ylabel = {$v(t)$},
              xmin = 0,
              xmax = 30,
              ymin = 0,
              ymax = 240,
              xtick = {0,10,20,30},
              xticklabels = {0,10,20,30},
              ytick = {0, 120},
              yticklabels = {0, 120},
              ]
              \addplot [
                domain=0:30,
                samples=2,
                color=blue,
                ]
                {120};
            \end{axis}
          \end{tikzpicture}
          \caption{}
        \end{figure}

        \begin{figure}[h]
          \centering
          \begin{tikzpicture}
            \begin{axis}[
              axis lines = left,
              xlabel = {$t$ ks},
              ylabel = {$P(t)$},
              xmin = 0,
              xmax = 30,
              ymin = 0,
              ymax = 2400,
              xtick = {0,10,20,30},
              xticklabels = {0,10,20,30},
              ytick = {0,600,1200,1800},
              yticklabels = {0,600,1200,1800},
              ]
              \addplot [
                domain=0:10,
                samples=2,
                color=black,
                ]
                {60*x+1200};
              \addplot [
                domain=10:20,
                samples=2,
                color=black,
                ]
                {1800};
              \addplot [
                domain=20:30,
                samples=2,
                color=black,
                ]
                {-60*x+3000};
            \end{axis}
          \end{tikzpicture}
          \caption{}
        \end{figure}

      \item Using the sketch of the power, find the total energy delivered to the battery.

        \begin{align}
          E_{\text{total}} &= \int_{0}^{30k} P(t) \, dt \\
                           &= ( 30k\times 1200 ) \\
                           &+ ( 10k\times 600 ) \\
                           &+ ( 10k\times 600 \times \frac{1}{2} ) \\
                           &+ ( 10k\times 600 \times \frac{1}{2} ) \\
                           &= 4.8 \times 10^7 \text{ J}
        .\end{align}
    \end{enumerate}

\end{enumerate}

\end{document}
